% Options for packages loaded elsewhere
\PassOptionsToPackage{unicode,hidelinks,linkcolor=black}{hyperref}
\PassOptionsToPackage{hyphens}{url}
\PassOptionsToPackage{dvipsnames,svgnames,x11names}{xcolor}
%
\documentclass[]{ascelike-new}

\usepackage{amsmath,amssymb}
\usepackage{iftex}
\ifPDFTeX
  \usepackage[T1]{fontenc}
  \usepackage[utf8]{inputenc}
  \usepackage{textcomp} % provide euro and other symbols
\else % if luatex or xetex
  \usepackage{unicode-math}
  \defaultfontfeatures{Scale=MatchLowercase}
  \defaultfontfeatures[\rmfamily]{Ligatures=TeX,Scale=1}
\fi
\usepackage{lmodern}
\ifPDFTeX\else  
    % xetex/luatex font selection
\fi
% Use upquote if available, for straight quotes in verbatim environments
\IfFileExists{upquote.sty}{\usepackage{upquote}}{}
\IfFileExists{microtype.sty}{% use microtype if available
  \usepackage[]{microtype}
  \UseMicrotypeSet[protrusion]{basicmath} % disable protrusion for tt fonts
}{}
\makeatletter
\@ifundefined{KOMAClassName}{% if non-KOMA class
  \IfFileExists{parskip.sty}{%
    \usepackage{parskip}
  }{% else
    \setlength{\parindent}{0pt}
    \setlength{\parskip}{6pt plus 2pt minus 1pt}}
}{% if KOMA class
  \KOMAoptions{parskip=half}}
\makeatother
\usepackage{xcolor}
\setlength{\emergencystretch}{3em} % prevent overfull lines
\setcounter{secnumdepth}{5}
% Make \paragraph and \subparagraph free-standing
\makeatletter
\ifx\paragraph\undefined\else
  \let\oldparagraph\paragraph
  \renewcommand{\paragraph}{
    \@ifstar
      \xxxParagraphStar
      \xxxParagraphNoStar
  }
  \newcommand{\xxxParagraphStar}[1]{\oldparagraph*{#1}\mbox{}}
  \newcommand{\xxxParagraphNoStar}[1]{\oldparagraph{#1}\mbox{}}
\fi
\ifx\subparagraph\undefined\else
  \let\oldsubparagraph\subparagraph
  \renewcommand{\subparagraph}{
    \@ifstar
      \xxxSubParagraphStar
      \xxxSubParagraphNoStar
  }
  \newcommand{\xxxSubParagraphStar}[1]{\oldsubparagraph*{#1}\mbox{}}
  \newcommand{\xxxSubParagraphNoStar}[1]{\oldsubparagraph{#1}\mbox{}}
\fi
\makeatother


\providecommand{\tightlist}{%
  \setlength{\itemsep}{0pt}\setlength{\parskip}{0pt}}\usepackage{longtable,booktabs,array}
\usepackage{calc} % for calculating minipage widths
% Correct order of tables after \paragraph or \subparagraph
\usepackage{etoolbox}
\makeatletter
\patchcmd\longtable{\par}{\if@noskipsec\mbox{}\fi\par}{}{}
\makeatother
% Allow footnotes in longtable head/foot
\IfFileExists{footnotehyper.sty}{\usepackage{footnotehyper}}{\usepackage{footnote}}
\makesavenoteenv{longtable}
\usepackage{graphicx}
\makeatletter
\def\maxwidth{\ifdim\Gin@nat@width>\linewidth\linewidth\else\Gin@nat@width\fi}
\def\maxheight{\ifdim\Gin@nat@height>\textheight\textheight\else\Gin@nat@height\fi}
\makeatother
% Scale images if necessary, so that they will not overflow the page
% margins by default, and it is still possible to overwrite the defaults
% using explicit options in \includegraphics[width, height, ...]{}
\setkeys{Gin}{width=\maxwidth,height=\maxheight,keepaspectratio}
% Set default figure placement to htbp
\makeatletter
\def\fps@figure{htbp}
\makeatother
% definitions for citeproc citations
\NewDocumentCommand\citeproctext{}{}
\NewDocumentCommand\citeproc{mm}{%
  \begingroup\def\citeproctext{#2}\cite{#1}\endgroup}
\makeatletter
 % allow citations to break across lines
 \let\@cite@ofmt\@firstofone
 % avoid brackets around text for \cite:
 \def\@biblabel#1{}
 \def\@cite#1#2{{#1\if@tempswa , #2\fi}}
\makeatother
\newlength{\cslhangindent}
\setlength{\cslhangindent}{1.5em}
\newlength{\csllabelwidth}
\setlength{\csllabelwidth}{3em}
\newenvironment{CSLReferences}[2] % #1 hanging-indent, #2 entry-spacing
 {\begin{list}{}{%
  \setlength{\itemindent}{0pt}
  \setlength{\leftmargin}{0pt}
  \setlength{\parsep}{0pt}
  % turn on hanging indent if param 1 is 1
  \ifodd #1
   \setlength{\leftmargin}{\cslhangindent}
   \setlength{\itemindent}{-1\cslhangindent}
  \fi
  % set entry spacing
  \setlength{\itemsep}{#2\baselineskip}}}
 {\end{list}}
\usepackage{calc}
\newcommand{\CSLBlock}[1]{\hfill\break\parbox[t]{\linewidth}{\strut\ignorespaces#1\strut}}
\newcommand{\CSLLeftMargin}[1]{\parbox[t]{\csllabelwidth}{\strut#1\strut}}
\newcommand{\CSLRightInline}[1]{\parbox[t]{\linewidth - \csllabelwidth}{\strut#1\strut}}
\newcommand{\CSLIndent}[1]{\hspace{\cslhangindent}#1}

\usepackage{booktabs}
\usepackage{longtable}
\usepackage{array}
\usepackage{multirow}
\usepackage{wrapfig}
\usepackage{float}
\usepackage{colortbl}
\usepackage{pdflscape}
\usepackage{tabu}
\usepackage{threeparttable}
\usepackage{threeparttablex}
\usepackage[normalem]{ulem}
\usepackage{makecell}
\usepackage{xcolor}
\usepackage[figurename=Fig.,labelfont=bf,labelsep=period]{caption}
\usepackage{newtxtext,newtxmath}
\makeatletter
\@ifpackageloaded{bookmark}{}{\usepackage{bookmark}}
\makeatother
\makeatletter
\@ifpackageloaded{caption}{}{\usepackage{caption}}
\AtBeginDocument{%
\ifdefined\contentsname
  \renewcommand*\contentsname{Table of contents}
\else
  \newcommand\contentsname{Table of contents}
\fi
\ifdefined\listfigurename
  \renewcommand*\listfigurename{List of Figures}
\else
  \newcommand\listfigurename{List of Figures}
\fi
\ifdefined\listtablename
  \renewcommand*\listtablename{List of Tables}
\else
  \newcommand\listtablename{List of Tables}
\fi
\ifdefined\figurename
  \renewcommand*\figurename{\textbf{Fig.}}
\else
  \newcommand\figurename{\textbf{Fig.}}
\fi
\ifdefined\tablename
  \renewcommand*\tablename{\textbf{Table}}
\else
  \newcommand\tablename{\textbf{Table}}
\fi
}
\@ifpackageloaded{float}{}{\usepackage{float}}
\floatstyle{ruled}
\@ifundefined{c@chapter}{\newfloat{codelisting}{h}{lop}}{\newfloat{codelisting}{h}{lop}[chapter]}
\floatname{codelisting}{Listing}
\newcommand*\listoflistings{\listof{codelisting}{List of Listings}}
\makeatother
\makeatletter
\makeatother
\makeatletter
\@ifpackageloaded{caption}{}{\usepackage{caption}}
\@ifpackageloaded{subcaption}{}{\usepackage{subcaption}}
\makeatother
\ifLuaTeX
  \usepackage{selnolig}  % disable illegal ligatures
\fi
\usepackage{bookmark}

\IfFileExists{xurl.sty}{\usepackage{xurl}}{} % add URL line breaks if available
\urlstyle{same} % disable monospaced font for URLs
\hypersetup{
  pdftitle={A utility-based approach to modeling systemic resilience of highway networks with an application in Utah},
  pdfauthor={Gregory S. Macfarlane; Max Barnes; Natalie M. Gray},
  pdfkeywords={Accessibility,, Location Choice,, Resiliency},
  colorlinks=true,
  linkcolor={black},
  filecolor={Maroon},
  citecolor={black},
  urlcolor={black},
  pdfcreator={LaTeX via pandoc}}

\title{A utility-based approach to modeling systemic resilience of
highway networks with an application in Utah}
\author[1*]{Gregory S. Macfarlane}%
\affil[1*]{
	Brigham Young University, Civil and Construction Engineering
Department; 
	Email: gregmacfarlane@byu.edu
	; Corresponding author}
\author[2]{Max Barnes}%
\affil[2]{
	Kimley-Horn; 
	Email: maxbarnes@kha.com
	}
\author[3]{Natalie M. Gray}%
\affil[3]{
	WSP; 
	Email: nat.gray2000@gmail.com
	}

\NameTag{Macfarlane, \today}
\begin{document}
\maketitle

\begin{abstract}
	The resilience of transportation networks is an important consideration
in policy, management and planning, but practical techniques to identify
systemically critical links are limited. Further, current practical
techniques ignore that when transportation networks are damaged or
degraded, people potentially change destinations and modes as well as
travel routes. In this research, we develop a model to examine network
highway resilience based on changes to mode and destination choice
logsums, and apply this model to 41 scenarios representing the loss of
links on the statewide highway network in Utah. The results of the
analysis suggest a fundamentally different prioritization scheme than
would be identified solely through a methodology based on increased
travel times. Beyond this, the comparable user costs of the logsum
method are generally lower than those considering only the value of
increased travel times.
\end{abstract}

\KeyWords{Accessibility,Location Choice,Resiliency}
\par
\vspace{1em}

\bookmarksetup{startatroot}

\section{Introduction}\label{intro}

Transportation agencies are tasked with maintaining effective transport
networks connecting a seemingly infinite variety of activities and
destinations. These networks are under constant obvious and non-obvious
threats, while the agencies possess finite resources with which to guard
against the threats. Identifying which facilities in a transportation
network are most \emph{critical} --- which would cause the most
disruption if they were to be degraded --- however, is not trivial.
State of the practice techniques typically rely on traffic volumes
represented in terms of vehicles, trips, or freight value (e.g.
\citeproc{ref-aem2020}{AEM 2020}). But these methods ignore the fact
that some networks have alternate routes readily available, have
multiple well developed modes of transportation, or have redundant
destination locations. When a network link does break, routes, mode
choice, and destination choice often change as well. These responses
have been observed in real-world crisis events like the I-35 W bridge
collapse in Minneapolis and the I-85 bridge fire and collapse in Atlanta
(\citeproc{ref-hamedi2018}{Hamedi et al. 2018};
\citeproc{ref-xie2011}{Xie and Levinson 2011};
\citeproc{ref-zhu2010}{Zhu et al. 2010a};
\citeproc{ref-levinson2010}{b}).

In this research, we apply a destination choice accessibility model to
identify critical highway links in a statewide highway context. This
model is designed to capture the utility lost to individuals operating
on a degraded network who may choose a longer route, a different travel
mode, or a different destination. We then apply the model to a highway
network representing the state of Utah. The model structure permits a
more nuanced evaluation of link criticality than using traffic volume
alone or in conjunction with an increased travel time measure.

The paper proceeds as follows. First, a \hyperref[litreview]{literature
review} discusses previous attempts to identify critical links in
transportation networks. A \hyperref[methodology]{methodology} section
presents the model developed for this research and describes the
implementation process in Utah. A \hyperref[results]{results} section
describes the model output in detail for one scenario and then compares
and ranks the model outputs for an array of several dozen independent
scenarios. The paper \hyperref[conclusion]{concludes} with a discussion
of limitations and associated avenues for future research.

\bookmarksetup{startatroot}

\section{Literature}\label{litreview}

Systemic resilience is an important consideration for transportation
agencies, though specific definitions of ``resilience'' might vary under
different contexts. Some agencies and researchers see resiliency as
facility-level design and engineering that hardens the system against
failure (\citeproc{ref-bradley2007}{Bradley 2007};
\citeproc{ref-peeta2010}{Peeta et al. 2010}); others as an ability for
maintenance staff to rapidly restore service following catastrophe
(\citeproc{ref-zhang2016}{Zhang and Wang 2016}); and others as an
ability for a system to continue operating in degraded state
(\citeproc{ref-berdica2002}{Berdica 2002}; \citeproc{ref-ip2011}{Ip and
Wang 2011}). Regardless of the definition used, assessing the resilience
of a transportation network --- and addressing any potential shortfalls
--- requires a method to identify which links or facilities are most
critical to the smooth operation of the network.

In a groundbreaking theoretical article, Berdica
(\citeproc{ref-berdica2002}{2002}) attempted to identify, define and
conceptualize network ``vulnerability'' --- the complement of resilience
--- by envisioning analyses conducted with several vulnerability
performance measures including travel time, delay, congestion,
serviceability, and accessibility. She then defined vulnerability as the
level of reduced accessibility due to unfavorable operating conditions
on the network. A network vulnerability is therefore a critical link
whose loss would have the largest negative impacts on the people using
the network. In particular, the author identified a need for further
research toward developing a framework capable of investigating
reliability and critical links in transportation networks.

In this section we examine several attempts by numerous researchers to
do precisely this using various measures of network performance. It is
helpful to categorize the existing literature into three groups
(summarized in Table~\ref{tbl-authortable}) based on the overall
technique applied in the study. These groups include:

\begin{itemize}
\item
  \emph{Network connectivity}: How does damage to a network diminish the
  connectivity between network nodes?
\item
  \emph{Travel time analysis}: How much do shortest path travel times
  between origins and destinations increase on a damaged network?
\item
  \emph{Accessibility analysis} How easily can the population using the
  damaged network complete their daily activities?
\end{itemize}

\begin{table}

\caption{\label{tbl-authortable}Attempts to Evaluate Systemic
Resiliency}

\centering{

\centering
\begin{tabular}[t]{rll}
\toprule
Year & Author & Performance Metric\\
\midrule
2004 & Geurs and van Wee & Accessibility (isochrone, gravity, logsum)\\
2006 & Scott et al. & Travel time and cost\\
2007 & Abdel-Rahim et al. & Network Connectivity\\
2008 & Taylor, M & Accessibility (logsum)\\
2010 & Peeta et al. & Travel time and cost\\
\addlinespace
2010 & Geurs et al. & Accessibility (logsum)\\
2010 & Levinson and Zhu & Travel time and cost\\
2010 & Zhu et al. & Travel time and cost\\
2011 & Agarwal et al. & Network connectivity\\
2011 & Ip and Wang & Network connectivity\\
\addlinespace
2011 & Serulle et al. & Travel time and cost\\
2011 & Ibrahim, S & Travel time and cost\\
2011 & Xie and Levinson & Accessibility (isochrone)\\
2012 & Jenelius and Mattson & Travel time and cost\\
2012 & Taylor and Susilawati & Accessibility (gravity)\\
\addlinespace
2013 & Omer et al. & Travel time and cost\\
2014 & Balijepalli and Oppong & Travel time and cost\\
2014 & Osei-Asamoah and Lownes & Network connectivity\\
2014 & Guze & Network connectivity\\
2015 & Zhang et al. & Network connectivity\\
\addlinespace
2015 & Jaller et al. & Travel time and cost\\
2015 & Xu et al. & Network connectivity\\
2016 & Winkler, C. & Accessibility (gravity)\\
2017 & Ganin et al. & Accessibility (gravity)\\
2019 & Vodak et al. & Network connectivity\\
\addlinespace
2019 & Hackl and Adey & Network connectivity\\
2019 & Gecchele et al. & Accessibility (logsum)\\
\bottomrule
\end{tabular}

}

\end{table}%

The purpose of transportation networks is to connect locations to each
other; presumably damage to a network would diminish the network's
\emph{connectivity}, or the number of paths between node pairs. It may
even leave nodes or groups of nodes completely isolated. In studies
using connectivity as the primary performance measure, researchers
typically apply methods and concepts from graph theory. These measures
may include elementary measures such as the isolation of nodes in a
network (\citeproc{ref-abdel2007}{Abdel-Rahim et al. 2007}). More
advanced measures have included heirarchical clustering of node paths
(\citeproc{ref-agarwal2011}{Agarwal et al. 2011};
\citeproc{ref-zhang2015}{Zhang et al. 2015}), a count of independent
paths (\citeproc{ref-vodak2019}{Vodák et al. 2019}), the reduction of
total network capacity (\citeproc{ref-ip2011}{Ip and Wang 2011}), and
special applications of the knapsack and traveling salesman problems
(\citeproc{ref-guze2014}{Guze 2014};
\citeproc{ref-osei2014}{Osei-Asamoah and Lownes 2014}). Though useful
from a theoretical perspective, many of these authors reported that
their approaches tend to break down to some degree on large, real-world
networks where the number of nodes and links numbers in the tens of
thousands, and the degree of connectivity between any arbitrary node
pair is high. They also do not typically account for how network users
may react to the new topology or capacity constraints of the degraded
network.

Highway system network failures --- in most imaginable cases --- degrade
the shortest or least cost path, but typically do not eliminate all
paths. The degree to which travel time increases when a particular link
is damaged, however, could provide an estimate of the criticality of
that link or node. This general method has been used to evaluate
potential choke points in various networks
(\citeproc{ref-berdica2007}{Berdica and Mattsson 2007};
\citeproc{ref-ganin2017}{Ganin et al. 2017};
\citeproc{ref-jaller2015}{Jaller et al. 2015};
\citeproc{ref-jenelius2012}{Jenelius and Mattsson 2012}) as well as the
allocation of emergency resources (\citeproc{ref-peeta2010}{Peeta et al.
2010}). Though many applications only consider the increase in travel
time, some authors consider how the users of the network will respond to
the decreased capacity (\citeproc{ref-balijepalli2014}{Balijepalli and
Oppong 2014}; \citeproc{ref-ibrahim2011}{Ibrahim et al. 2011};
\citeproc{ref-scott2006}{Scott et al. 2006};
\citeproc{ref-serulle2011}{Serulle et al. 2011};
\citeproc{ref-xu2015}{Xu et al. 2015}), and others attempt to model a
shift in departure time or mode (\citeproc{ref-omer2013}{Omer et al.
2013}).

A primary limitation with increased travel time methodologies is that
they ignore other possible ways a population might adapt its travel to a
damaged network. Aside from shifting routes and modes, people may choose
other destinations and it is possible that some previously planned trips
might be canceled entirely. Travel time-based methods do not account for
the costs of these changes in plans, when people select (presumably
worse) activity locations and modes of travel in addition to seeing
their travel costs increase because of the additional travel time. But
accessibility methods --- in particular the accessibility calculations
embedded within many existing regional transport models ---- provide a
framework for evaluating these costs
(\citeproc{ref-ben-akiva1985}{Ben-Akiva and Lerman 1985};
\citeproc{ref-geurs2004}{Geurs and van Wee 2004}).

Accessibility is an abstract concept with multiple methods of
quantification (\citeproc{ref-handy1997}{Handy and Niemeier 1997}).
Perhaps the most popular method is the cumulative opportunities measure:
e.g., the number of jobs within a specified travel time threshold. This
is the method employed by Xie and Levinson
(\citeproc{ref-xie2011}{2011}) in an analysis of the impact of the I-35
W bridge collapse in Minneapolis, and in a theoretical context in
Australia by Taylor and Susilawati (\citeproc{ref-taylor2012}{2012}).
But cumulative opportunities measures require the analyst to assert a
travel time threshold, a mode, and an opportunity of interest. Some of
these assumptions can be relaxed with a gravity-style access model, but
the logsum of a destination choice model has several benefits as an
accessibility term including its grounding in choice framework, ability
to weigh multiple attributes of an alternative, and include travel
impedances by all modes (\citeproc{ref-dong2006}{Dong et al. 2006}).
These measures can even be weighted by a cost coefficient to translate
the lost utility in monetary terms (\citeproc{ref-geurs2010}{Geurs et
al. 2010}; \citeproc{ref-geurs2004}{Geurs and van Wee 2004}).

Logsum-derived accessibility meaures have been used before to evaluate
network resiliency (\citeproc{ref-masiero2012}{Masiero and Maggi 2012};
\citeproc{ref-taylor2008}{Taylor 2008};
\citeproc{ref-winkler2016}{Winkler 2016}). Miller et al.
(\citeproc{ref-miller2015}{2015}) and Gecchele et al.
(\citeproc{ref-gecchele2019}{2019}), for example, each employ an
activity-based model to evaluate the change travel demand related to a
highway link --- implicitly using the log sums of a trip and destination
choice model --- but only evaluate the change in observed choices rather
than the implied costs of those choices at the utility level. But beyond
this, these research efforts have not yet entered the mainstream in
practical application. Though previous researchers have shown that
logsum-derived accessibility measures are feasible and informative,
their use in actual resilience analysis efforts by state departments of
transportation and other relevant agencies appears limited. This limited
practical application is unfortunate, given the important theoretical
steps described to this point.

\bookmarksetup{startatroot}

\section{Methodology}\label{methodology}

In this section we describe a model framework designed to evaluate
resilience using a logit-based choice metric. This framework is heavily
based on tools and methods in existing statewide travel models, with a
few necessary extensions. We then describe the implementation of this
model framework to a prioritization exercise on the Utah Statewide
Travel Model (USTM).

\subsection{Model Design}\label{model-design}

The overall model framework is presented in Fig.~\ref{fig-framework},
and is designed to capture the utility-based accessibility for a
particular origin zone \(i\) and trip purpose \(m\). The model begins
with a travel time skim procedure, to determine the congested travel
time from zone \(i\) to zone \(j\) by auto as well as the shortest
network distance for non-motorized modes. The transit travel time skim
is fixed, assuming that transit infrastructure would not be affected by
changes to the highway network. Throughout this section, lower-cased
index variables \(k\) belong to a set of all indices described by the
corresponding capital letter \(K\).

\begin{figure}

\centering{

\includegraphics[width=1\textwidth,height=\textheight]{03-methods_files/figure-pdf/fig-framework-1.png}

}

\caption{\label{fig-framework}Model framework with feedback cycle. Blue
boxes are calculated after the second feedback loop.}

\end{figure}%

With the travel time \(t_{ijk}\) for all modes \(k \in K\), the model
computes mode choice utility values. The multinomial logit mode choice
model describes the probability of a person at origin \(i\) choosing
mode \(k\) for a trip to destination \(j\):
\begin{equation}\phantomsection\label{eq-mcp}{
\mathcal{P}_{ijm}(k) = \frac{\exp(f(\beta_{m}, t_{ijk}))}{\sum_{K}\exp(f(\beta_{m}, t_{ijk}))}
}\end{equation} The log of the denominator of the this equation is
called the mode choice logsum, \(MCLS_{ijm}\) and is a measure of the
travel cost by all modes, weighted by utility parameters \(\beta_m\)
that may vary by trip purpose.

The \(MCLS\) is then used as a travel impedance term in the multinomial
logit destination choice model, where the probability of a person at
origin \(i\) choosing destination \(j \in J\) is
\begin{equation}\phantomsection\label{eq-dcp}{
\mathcal{P}_{im}(j) = \frac{\exp(f(\gamma_{m}, MCLS_{ijm}, A_j))}{\sum_{J}\exp(f(\gamma_{m}, MCLS_{ijm}, A_j))}
}\end{equation} where \(A_j\) is the attractiveness --- represented in
terms of socioeconomic activity --- of zone \(j\). As with mode choice,
the log of the denominator of this model is the destination choice
logsum, \(DCLS_{im}\). This quantity represents the value access to all
destinations by all modes of travel, and varies by trip purpose.

The \(DCLS_{im}\) measure is relative, but can be compared across
scenarios. The difference between the measures of two scenarios
\begin{equation}\phantomsection\label{eq-deltas}{
\Delta_{im} = DCLS_{im}^{\mathrm{Base}} - DCLS_{im}^{\mathrm{Scenario}}
}\end{equation} provides an estimate of the accessibility lost when
\(t_{ij\mathrm{drive}}\) changes due to a damaged highway link. This
accessibility change is \emph{per trip}, meaning that the total lost
accessibility is \(P_{im} * \Delta_{im}\) where \(P\) is the number of
trip productions at zone \(i\) for purpose \(m\). This measure is given
in units of dimensionless utility, but the mode choice cost coefficient
\(\beta_{\mathrm{cost}}\) provides a conversion factor between utility
and cost. The total financial cost of a damaged link for the entire
region for all trip purposes is
\begin{equation}\phantomsection\label{eq-totalcost}{
\mathrm{Cost} = \sum_{I}\sum_{M} -1 / \beta_{\mathrm{cost},m} * P_{im} \Delta_{im}
}\end{equation}

For comparison to a common method that only includes the increased
travel time between origins and destinations (and not the cost and
opportunities of changing modes and destinations), we compute the change
in congested travel time between \(\delta t_{ij}\) and multiply the
number of trips by this change and a value of time coefficient derived
from the cost and vehicle time coefficients of the mode choice model,
\begin{equation}\phantomsection\label{eq-ttmethod}{
\mathrm{Cost}' =  \sum_I \sum_J \sum_M \frac{\beta_{\mathrm{time}, m} }{\beta_{\mathrm{cost}, m}} T_{ijm} \delta t_{ijm}
}\end{equation}

\subsection{Model Implementation in
Utah}\label{model-implementation-in-utah}

The Utah Department of Transportation (UDOT) manages an extensive
highway network consisting of interstate freeways (I-15, I-80, I-70, and
I-84), intraurban expressways along the Wasatch Front, and rural
highways throughout the state. The rugged mountain and canyon topography
throughout the state places severe constraints on possible redundant
paths in the highway network. A landslide or rock fall in any single
canyon may isolate a community or force a redirection of traffic that
could be several hours longer than the preferred route; understanding
which of these many possible choke points is most critical is a key and
ongoing objective of the agency.

Several data elements for the model described above were obtained from
the Utah Statewide Travel Model (USTM). USTM is a trip-based statewide
model that is focused exclusively on long-distance and rural trips:
intraurban trips within existing Metropolitan Planning Organization
(MPO) model regions are pre-loaded onto the USTM highway network. This
means that USTM as currently constituted can be used for infrastructure
planning purposes, but would be inadequate to evaluate the systemic
resiliency of the highway network given the disparate methodologies of
the MPO models. USTM can, however, provide the following data elements

\begin{enumerate}
\def\labelenumi{\arabic{enumi}.}
\tightlist
\item
  \emph{Highway Network}: including free flow and congested travel
  speeds, link length, link capacity estimates, etc.
\item
  \emph{Zonal Productions} \(P_{im}\): available for all zones by
  purpose, including those in the MPO region areas.
\item
  \emph{Zonal Socioeconomic Data}: the destination choice model
  described in Eq.~\ref{eq-dcp} calculates attractions \(A_{jm}\) from
  the USTM zonal socioeconomic data based on the utility coefficients in
  Table~\ref{tbl-coeffs}.
\item
  \emph{Calibration Targets}: USTM base scenario estimates of mode split
  and trip length were used to calibrate the utility coefficients as
  described below.
\end{enumerate}

Among MPO models in Utah, only the model jointly operated by the Wasatch
Front Regional Council (WFRC, Salt Lake area MPO) and the Mountainland
Association of Governments (MAG, Provo area MPO) model includes a
substantive transit forecasting component. The transit travel time skim
from the WFRC / MAG model was used for the mode choice model in
Eq.~\ref{eq-mcp}; the zonal travel time between the smaller WFRC / MAG
model zones was averaged to the larger USTM zones, and the minimum time
among the several modes available (commuter rail, light rail, bus rapid
transit, local bus) was taken as the travel time for a single transit
mode in this implementation.

\begin{table}

\caption{\label{tbl-coeffs}Choice Model Coefficients}

\centering{

\centering
\begin{tabular}[t]{llrrr}
\toprule
 & Variable & HBW & HBO & NHB\\
\midrule
\addlinespace[0.3em]
\multicolumn{5}{l}{\textbf{Destination Choice}}\\
\hspace{1em} & Households & 0.0000 & 1.0187 & 0.2077\\

\hspace{1em} & Office Employment & 0.4568 & 0.4032 & 0.2816\\

\hspace{1em} & Other Employment & 1.6827 & 0.4032 & 0.2816\\

\hspace{1em} & Retail Employment & 0.6087 & 3.8138 & 5.1186\\

\hspace{1em} & Distance & -0.0801 & -0.1728 & -0.1157\\

\hspace{1em} & Distance\textasciicircum{}2 & 0.0026 & 0.0034 & 0.0035\\

\hspace{1em} & Distance\textasciicircum{}3 & 0.0000 & 0.0000 & 0.0000\\
\cmidrule{1-5}
\addlinespace[0.3em]
\multicolumn{5}{l}{\textbf{Mode Choice}}\\
\hspace{1em} & Shared & -1.1703 & 0.0164 & -0.0336\\

\hspace{1em} & Transit & -0.3903 & -1.9811 & -2.2714\\

\hspace{1em} & Non-Motorized & -1.2258 & -0.3834 & -0.8655\\

\hspace{1em} & Travel Time [minutes] & -0.0450 & -0.0350 & -0.0400\\

\hspace{1em} & Travel Cost [dollars] & -0.0016 & -0.0016 & -0.0016\\

\hspace{1em} & Walk Distance (less than 1 mile) [miles] & -0.0900 & -0.0700 & -0.0800\\

\hspace{1em} & Walk Distance (1 mile or more) [miles] & -0.1350 & -0.1050 & -0.1200\\
\bottomrule
\end{tabular}

}

\end{table}%

The utility coefficients for the destination and mode choice models are
presented in Table~\ref{tbl-coeffs}. The mode choice coefficients were
adapted from USTM and supplemented with coefficients from the Roanoke
(Virginia) Valley Transportation Planning Organization (RVTPO) travel
model (\citeproc{ref-wang2016}{Virginia Department of Transportation
2016}). This model was selected for its simplicity and analogous data
elements to the proposed model. The alternative-specific constants were
calibrated to regional mode choice targets developed from the 2012 Utah
Household Travel Survey (UHTS) using methods described by Koppelman and
Bhat (\citeproc{ref-koppelman2006}{2006}).

The destination choice utility equation consists of three parts: a size
term, a travel impedance term, and a calibration polynomial.
Coefficients for the size term and travel impedance terms were adapted
from the Oregon Statewide Integrated Model (SWIM)
(\citeproc{ref-donnelly2017}{Donnelly 2017}) for all purposes except
HBW, which coefficients were adapted from the RVTPO model as the SWIM
uses a different methodology for selecting work locations. The distance
polynomial coefficients were calibrated to targets developed from the
2012 UHTS.

\subsubsection{Vulnerable Link
Identification}\label{vulnerable-link-identification}

To develop evaluation scenarios on which to apply the model, we used
information contained in the UDOT Risk Priority Analysis online map
(\citeproc{ref-UDOT2020}{UDOT 2020}). This map considers the probability
of various events that could impact road performance including rock
falls, avalanches, landslides, and other similar occurrences. Using this
tool, combined with information gathered from the research team and UDOT
officials, 41 locations of interest were identified for analysis, at the
locations shown in Fig.~\ref{fig-linksmap}. Each link was identified due
to its location in relation to population centers, remote geographic
location, and proximity to other highway facilities, or because the link
was known to be at risk due to geologic or geographic features, or
because it was a suspected choke point in the network.

\bookmarksetup{startatroot}

\section{Results}\label{sec-results}

In this section we apply the model to scenarios where critical highway
links are removed from the model network. This includes first a detailed
analysis of a single scenario, where I-80 between Salt Lake and Tooele
Counties is severed. We compare the model output to an alternative
method that measures only the change in travel time and does not allow
for mode or destination choice. The model was then applied to 40
additional link closure scenarios throughout the state.

\subsection{Detailed Scenario
Analysis}\label{detailed-scenario-analysis}

To illustrate how the logsum-based model framework captures the costs of
losing a link, we conducted a scenario where I-80 between Tooele and
Salt Lake Counties west of the Salt Lake City International Airport
becomes unavailable. Tooele is a largely rural county with increasing
numbers of residents who commute to jobs in the Salt Lake Valley. I-80
is the only realistic path between the two counties, as alternate routes
involve mountain canyons and many additional miles of travel.

The costs obtained by the logsum and travel time based methods for this
scenario are shown in Table~\ref{tbl-tooeletable}.
Fig.~\ref{fig-tooelemap} presents the lost HBW utility in each TAZ
associated with removing the interstate link between Tooele and Salt
Lake Counties. Multiplying these values by the population in each
production TAZ, summing based on whether the TAZ is inside or outside
Tooele County, and then multiplying by the cost coefficient yields the
HBW values for the Utility Logsum in Table~\ref{tbl-tooeletable}.

In both the logsum and travel time analyses, we separate the costs
spatially, though the definitions of the two are slightly different. In
the logsum-based method, ``Inside Tooele'' are costs associated with
decreased accessibility for trips produced in Tooele County. The
increased costs in this case capture the loss in utility by measuring
increased multi-model travel impedances to destinations with high
attractiveness. In the travel-time method, by contrast, the ``Inside
Tooele'' costs are those for trips with an origin in Tooele County and a
destination in Salt Lake County, and are the increase in travel time
multiplied by a value of time and the number of vehicles making the
trip. The difference in definition is required by the difference in
framework construction.

In general, the logsum-based method arrives at cost estimates that are
less than half of the comparable estimates of the travel time-based
method. This is not unexpected, as the travel time-based method assumes
that all the preexisting trips would still occur, but on a longer path.
The logsum-based method on the other hand attempts to capture the fact
that when a path changes, people may shift their destination or their
mode of travel. To be clear, the framework we have developed for this
exercise does not explicitly model the selection of a new alternative
destination; rather, we calculate instead the value of a destination
choice set before and after the link is removed. But the implication is
that the availability of alternative destinations still provide some
benefit to the choice maker, a proposition that the travel time method
ignores.

\begin{table}

\caption{\label{tbl-tooeletable}Comparison of Logsum-based and
Time-based costs for I-80 at Tooele}

\centering{

\centering
\resizebox{\linewidth}{!}{
\begin{tabular}[t]{lrrrrrr}
\toprule
\multicolumn{1}{c}{ } & \multicolumn{3}{c}{Utility Logsum} & \multicolumn{3}{c}{Travel Time} \\
\cmidrule(l{3pt}r{3pt}){2-4} \cmidrule(l{3pt}r{3pt}){5-7}
Purpose & Other Counties & Inside Tooele & Total & Other Counties & Inside Tooele & Total\\
\midrule
\addlinespace[0.3em]
\multicolumn{7}{l}{\textbf{Passenger}}\\
\hspace{1em}HBO & \$2,637 & \$28,290 & \$30,927 & \$8,595 & \$75,862 & \$84,457\\
\hspace{1em}HBW & \$4,039 & \$113,660 & \$117,699 & \$7,868 & \$234,009 & \$241,877\\
\hspace{1em}NHB & \$2,141 & \$44,911 & \$47,051 & \$5,875 & \$100,245 & \$106,120\\
\addlinespace[0.3em]
\multicolumn{7}{l}{\textbf{External / Freight}}\\
\hspace{1em}Internal Freight &  &  &  & \$24,617 & \$26,083 & \$50,700\\
\hspace{1em}Inbound / Outbound Freight &  &  &  & \$59,758 & \$1,190 & \$60,948\\
\hspace{1em}Recreation &  &  &  & \$176 & \$190 & \$366\\
\hspace{1em}Through Freight &  &  &  & \$251,508 &  & \$251,508\\
\hspace{1em}Through Passenger &  &  &  & \$56,103 &  & \$56,103\\
Comparable Total & \$8,817 & \$186,860 & \$195,677 & \$22,338 & \$410,116 & \$432,454\\
Total & \$8,817 & \$186,860 & \$195,677 & \$414,499 & \$437,580 & \$852,079\\
\bottomrule
\end{tabular}}

}

\end{table}%

\begin{figure}

\centering{

\includegraphics{04-results_files/figure-pdf/fig-tooelemap-1.pdf}

}

\caption{\label{fig-tooelemap}Tooele county HBW utility loss}

\end{figure}%

Another key element of Table~\ref{tbl-tooeletable} is that the largest
single element of costs in the scenario is associated with through
freight, as well as contributions from other purposes for which the
logsum model developed in this study did not include a corresponding
methodology. This was due to data limitations and the modeling approach
of the existing USTM, but it is obvious that a choice-based framework
for examining the costs of through and inbound / outbound traffic is an
important limitation in this scenario and potentially many others.

\subsection{Prioritization Scenario
Results}\label{prioritization-scenario-results}

We now apply the model to compare 40 additional scenarios where
individual highway facilities are removed from the model highway
network. Fig.~\ref{fig-traveltimerank} presents the logsum and travel
time results for all 41 scenarios, ordered by decreasing logsum costs.
In other words, I-15 at the boundary between Utah and Salt Lake Counties
is the most ``critical'' link analyzed in this exercise as determined by
the logsum method. Were this link to be cut, the people of Utah would
have the highest costs per day in lost destination and travel access of
any other link. Each of the highest-ranking roads in this analysis is an
interstate facility in northern Utah, which is more heavily populated
than the remote areas in the south. A map showing the locations of these
scenarios is given in Fig.~\ref{fig-linksmap}. The concentration of the
highest cost links in the Salt Lake City metropolitan area is obvious,
though the links with the highest cost are not \emph{in} Salt Lake City
where multiple parallel routes exist. Rather, they are in mountain
canyons surrounding the main urban area.

\begin{figure}

\begin{minipage}{\linewidth}

\centering{

\includegraphics{04-results_files/figure-pdf/fig-linksmap-1.pdf}

}

\subcaption{\label{fig-linksmap-1}Statewide}

\end{minipage}%
\newline
\begin{minipage}{\linewidth}

\centering{

\includegraphics{04-results_files/figure-pdf/fig-linksmap-2.pdf}

}

\subcaption{\label{fig-linksmap-2}Wasatch Front}

\end{minipage}%

\caption{\label{fig-linksmap}Total cost of link closure for each
scenario by the logsum method.}

\end{figure}%

Perhaps strangely, many scenarios in the analysis show a \emph{benefit}
from loss of the link. Investigating these scenarios showed that for
many paths, the shortest automobile travel time in the complete network
is \emph{not} the shortest path by distance. When the shortest time path
is disrupted, the new shortest time path may be only a few minutes
longer by time but dozens of miles shorter (on a slower road). Because
the automobile operating costs are calculated per mile and not per
minute, this means that the new path actually produces a benefit. This
challenge is exacerbated by the apparent agency practice of placing
artificial time penalties on the network links in some canyons during
calibration. This exercise reveals one reason why such a practice should
be discouraged, and also highlights the importance of using consistent
functions for impedance calculation at all stages of the model.

\begin{figure}

\centering{

\includegraphics{04-results_files/figure-pdf/fig-traveltimerank-1.pdf}

}

\caption{\label{fig-traveltimerank}Scenario results of travel time and
logsum methods in comparison.}

\end{figure}%

Fig.~\ref{fig-traveltimerank} also presents the costs associated with
link closure based on the travel time method for comparable purposes and
also for freight. Many of the most costly scenarios in the logsum model
also appear to be costly in the comparable elements of the travel time
method. That is, the scenario breaking the interstate link between Salt
Lake and Utah counties is the most costly scenario in both methods, and
underscores the significance of this link to Utah's economy and people.
But while many of the largest and most impactful scenarios have similar
rankings and scales, there are also drastic differences between the two
methods down the line. To put it simply, the choice of analysis method
would change the priority that UDOT places on its roads in terms of
preparing for incidents and hardening assets.

\subsection{Sensitivity Analysis}\label{sensitivity-analysis}

A primary limitation of the model framework presented to this point is
that the input parameters used for the mode and destination choice
utilities were gathered from several different sources including a
statewide trip-based model, a statewide activity-based model, and an
urban model for a small region. These models were transferred without a
knowledge of the individual parameter estimation statistics or overall
goodness of fit. How much would the findings presented to this point
change if the parameters in Table~\ref{tbl-coeffs} were to change
modestly?

To examine this possibility, we construct 25 independent draws of the
parameter coefficients using Latin Hypercube Sampling
(\citeproc{ref-helton2003}{Helton and Davis 2003}). The coefficient of
variation for each parameter was assumed to be 0.1; originally, a value
of 0.3 was selected based on the uncertainty research of Zhao and
Kockelman (\citeproc{ref-zhao2002}{2002}), but this resulted in an
unreasonable range of implied values of time in many parameter draws.
Using each of the 25 draws, we ran the base scenario and three
large-impact scenarios and calculated the logsum-based costs.

Fig.~\ref{fig-saplot} presents the estimated monetary costs for each of
these three scenarios under each of the 25 parameter draws. The results
are ordered in the figure by the estimated cost for the highest-impact
scenario. Two observations can be made from this figure. First,
different parameter values do not affect the scenarios uniformly. The
second observation is that despite the within-scenario variation, the
overall scale of the three scenarios is maintained regardless of the
drawn parameter values. Indeed, the three scenarios remain in their
priority ranking across all 25 draws of the choice model parameters. We
therefore do not expect that the selection of parameters is a major
element in the relative ranking of scenarios.

\begin{figure}

\centering{

\includegraphics{04-results_files/figure-pdf/fig-saplot-1.pdf}

}

\caption{\label{fig-saplot}Estimated logsum-based scenario costs in 25
different draws of the choice model parameters.}

\end{figure}%

\bookmarksetup{startatroot}

\section{Limitations and Discussion}\label{limitations-and-discussion}

The issue of systemic resilience to natural and other hazards is an
important question for transportation agencies in the United States and
around the world, given the precarious situation of many of its
transportation assets. Flash floods, rockfalls, avalanches, earthquakes,
and other incidents pose threats to the network of independent
transportation facilities. This research did not consider risk
assessment directly at any level, but rather took as a given that 41
facilities were at some level of risk and played a potentially large
systemic role. A well-conceived approach to systemic resilience should
involve all three elements of resilience: hardening assets from failure
through high-quality engineering and construction; locating maintenance
resources in areas where they can most quickly resolve issues and return
facilities to optimal conditions; and understanding how the system could
work effectively in a damaged or degraded state for medium to long
periods of time if necessary.

This research focused on the systemic criticality of 41 facilities that
were assumed to fail independently. Some of the disaster scenarios most
likely to affect highway facilities -- especially a major earthquake --
are likely to damage multiple highway assets simultaneously. This
research might be extended to consider what would happen if a set of
highway facilities failed; is there a facility that is not critical were
it to fail by itself, but ends up being a critical component of several
combinations of failures? Taking the question further, agencies might
consider scenarios where emergency services or evacuations must operate
on a degraded network, an element of the emerging research area of
functional recovery (\citeproc{ref-zhan2022}{Zhan et al. 2022}). Of
course, it is also an open question as to whether destination and mode
choices will follow the same behavioral patterns in such a scenario.

It is also worth considering that the behavioral response when links in
a highway network are degraded is directly analogous with the behavioral
response when network links are \emph{improved.} As travel model
methodologies improve to meet the challenges posed by induced demand
(\citeproc{ref-thill2005}{Thill and Kim 2005};
\citeproc{ref-volker2020}{Volker et al. 2020}), it would seem strange to
use one methodology for positive network improvements while inventing
wholly new methodologies for negative network improvements.

This research developed a trip-based statewide transportation planning
model using common model design practices including a destination choice
trip distribution model and a complete ---- if rudimentary ----
logit-based mode choice model. Many agencies use a trip-based statewide
transportation planning model with a basic gravity-type trip
distribution model and no mode choice component that cannot be used
directly for this kind of analysis. Using logit-based choice frameworks
for trip distribution and mode choice allows the model to incorporate
greater sensitivity to influencing variables and other benefits, in
addition to providing data to support this type of resiliency analysis.
The model presented in this particular research should not be used for
infrastructure policy or forecasting analyses as its parameters have
been transferred from three different source models and not estimated
from local data directly. Rather, the model developed for this research
is an illustrative tool, with some degree of internal consistency and
robustness as illustrated by the parameter sensitivity analysis.
Agencies should consider the full range of policies that their models
could evaluate when they make design decisions about how models are
structured.

On a related note, only one of the three source models was accompanied
with open documentation describing the estimation of the models along
with goodness of fit and significance statistics. Publishing model
estimation reports --- perhaps even alongside anonymized datasets to
allow re-estimation --- would strengthen the scientific basis of travel
forecasting and help earn trust among the public. This practice would
have allowed the sensitivity analysis presented in this report to be
based on draws from the covariance matrix of the individual parameters
rather than an asserted coefficient of variation.

The flexibility and sensitivity afforded by choice models can introduce
some additional challenges, however. The results of the logsum-based
choice analysis highlight one such risk, in that using different cost
functions for network skimming and destination choice can lead to
inconsistent model behavior. In this research and in the USTM highway
assignment --- and common to many agency modeling practices --- vehicle
trips between origins and destinations are loaded onto the shortest time
path, but their destinations are chosen by a combination of travel time
and path distance. Though this issue does not always arise during
calibration and typical volume forecasting efforts, it represents a
serious inconsistency in the travel model framework. Along the same
lines, this research included only a single feedback iteration;
understanding how many iterations are necessary for a travel model to
successfully converge -- where the destinations chosen are no longer
changing based on changes to travel times between origins and
destinations -- is an important model design decision that was not
explored here.

A potential limitation in the model developed for this research is that
all HBW trips are flexible in destination choice. This implies that a
user could choose to work in a different place when the path to their
previously-chosen work location is disrupted. This might not be entirely
logical for short-term or even medium-term highway closures, considering
that most people will not switch jobs so quickly. With the recent
increase in telecommuting instigated by the COVID-19 pandemic, workplace
location is likely even more flexible now than it has been in the past.
This increased flexibility has already called into question how HBW
trips are handled in travel behavior modeling
(\citeproc{ref-salon2021}{Salon et al. 2021}). A more nuanced method for
estimating HBW trips that accounts for both flexibility and
inflexibility of workplace location might be developed, or the use of
activity-based models that consider journeys to work or telecommuting as
a function of trip distance could be employed.

A similar limitation of this research is that we held the travel times
and levels of service constant for non-highway modes. While some
possible incidents would affect only highways --- such as losing one
interstate bridge --- many would affect adjacent rail transit
infrastructure, or would disrupt non-motorized paths or local transit
services which use the highways. Ignoring the potential disruption to
these modes potentially leads our model to underestimate the user costs
for some scenarios where alternative modes would not be available.

Another limitation in this research surrounds the daily trip assignment
procedure, which introduces two related issues. The first is that the
analysis does not consider incidents which close a facility for a time
shorter than 24 hours. Though the target of this research was an
understanding of incidents of longer duration (allowing for a shift in
destination choice), the impacts of shorter incidents on user costs is
an important topic. A departure time choice model might allow users to
shift trips to periods with a less compromised network. Addressing this
issue, however, would introduce challenges related to long-distance
trips that would stretch across time periods. A dynamic traffic
assignment or mesoscopic network simulation may be a better strategy to
address this issue (e.g., \citeproc{ref-kaddoura2018}{Kaddoura and Nagel
2018}).

Policies that result in a clear and certain outcome are rare, though
travel demand models are sometimes misinterpreted, misused, or even
mis-designed to imply a single policy prediction. Along the same lines,
agencies should take a proactive role in helping travel modelers and
transportation planners incorporate uncertainty in their analysis and
convey this uncertainty responsibly in communications with
transportation decision-makers and the general public. The sensitivity
test supplied in this research is a modest step in this direction.

\bookmarksetup{startatroot}

\section{Conclusions}\label{conclusions}

The question of systemic resilience of highway assets is an increasing
concern to agencies that must maintain critical infrastructure as it
ages in the face of a changing climate and economy. The basic tools of
travel forecasting --- based on coherent representations of human
behavior --- provide compelling tools for evaluating the criticality of
individual highway assets. These tools may require different expertise
than more traditional agency engineers are usually equipped with, but
simplified or simplistic methods can lead to fundamentally different
evaluations of what may happen when the network changes. Specifically,
the results of this analysis suggest travel time based methods may
overstate the excess user costs associated with degraded highway
networks, as they do not consider that travelers might choose different
destinations or modes to accomplish their travel. Demand modelers must
first equip their models with the tools to be useful outside of simple
volume forecasting, and then communicate with their stakeholders and
peers the purpose of the model and its powerful potential for helpful
analysis.

\bookmarksetup{startatroot}

\section*{Acknowledgments}\label{acknowledgments}
\addcontentsline{toc}{section}{Acknowledgments}

\markboth{Acknowledgments}{Acknowledgments}

This study was funded by the Utah Department of Transportation. The
authors alone are responsible for the preparation and accuracy of the
information, data, analysis, discussions, recommendations, and
conclusions presented herein. The contents do not necessarily reflect
the views, opinions, endorsements, or policies of the Utah Department of
Transportation or the US Department of Transportation. The Utah
Department of Transportation makes no representation or warranty of any
kind, and assumes no liability therefore.

\bookmarksetup{startatroot}

\section*{Data Availability
Statement}\label{data-availability-statement}
\addcontentsline{toc}{section}{Data Availability Statement}

\markboth{Data Availability Statement}{Data Availability Statement}

Some or all data, models, or code that support the findings of this
study are available from the corresponding author upon reasonable
request.

\bookmarksetup{startatroot}

\section*{Author Contribution
Statement}\label{author-contribution-statement}
\addcontentsline{toc}{section}{Author Contribution Statement}

\markboth{Author Contribution Statement}{Author Contribution Statement}

\textbf{Gregory S. Macfarlane:} Conceptualization, Methodology, Writing
- review \& editing, Supervision \textbf{Max Barnes:} Methodology,
Software, Formal Analysis, Investigation, Data curation, Writing -
original draft \textbf{Natalie Gray:} Formal Analysis, Investigation,
Data curation, Writing - original draft, Visualization

\bookmarksetup{startatroot}

\section*{References}\label{references}
\addcontentsline{toc}{section}{References}

\markboth{References}{References}

\phantomsection\label{refs}
\begin{CSLReferences}{1}{0}
\bibitem[\citeproctext]{ref-abdel2007}
Abdel-Rahim, A., P. Oman, B. K. Johnson, and R. A. Sadiq. 2007.
{``\href{https://doi.org/10.1109/ITSC.2007.4357801}{Assessing surface
transportation network component criticality: A multi-layer graph-based
approach}.''} \emph{2007 IEEE intelligent transportation systems
conference}, 1000--1003.

\bibitem[\citeproctext]{ref-aem2020}
AEM. 2020. \emph{Risk and resilience analysis procedure}. Available at
https://www.codot.gov/programs/planning/assets/cdot-rnr-analysis-procedure-8-4-2020-v6.pdf.

\bibitem[\citeproctext]{ref-agarwal2011}
Agarwal, J., M. Liu, and D. Blockley. 2011.
{``\href{https://doi.org/10.1061/41170(400)28}{A systems approach to
vulnerability assessment}.''} \emph{Vulnerability, uncertainty, and
risk}, 230--237.

\bibitem[\citeproctext]{ref-balijepalli2014}
Balijepalli, C., and O. Oppong. 2014. {``Measuring vulnerability of road
network considering the extent of serviceability of critical road links
in urban areas.''} \emph{Journal of Transport Geography}, 39: 145--155.
\url{https://doi.org/10.1016/j.jtrangeo.2014.06.025}.

\bibitem[\citeproctext]{ref-ben-akiva1985}
Ben-Akiva, M., and S. R. Lerman. 1985. \emph{Discrete choice analysis:
Theory and applications to travel demand}. MIT Press.

\bibitem[\citeproctext]{ref-berdica2002}
Berdica, K. 2002. {``An introduction to road vulnerability: What has
been done, is done and should be done.''} \emph{Transport Policy}, 9
(2): 117--127. \url{https://doi.org/10.1016/S0967-070X(02)00011-2}.

\bibitem[\citeproctext]{ref-berdica2007}
Berdica, K., and L.-G. Mattsson. 2007. {``Vulnerability: A model-based
case study of the road network in stockholm.''} \emph{Critical
infrastructure}, 81--106. Springer.

\bibitem[\citeproctext]{ref-bradley2007}
Bradley, J. 2007. {``Time period and risk measures in the general risk
equation.''} \emph{Journal of Risk Research}, 10 (3): 355--369.
Routledge. \url{https://doi.org/10.1080/13669870701252232}.

\bibitem[\citeproctext]{ref-dong2006}
Dong, X., M. E. Ben-Akiva, J. L. Bowman, and J. L. Walker. 2006.
{``Moving from trip-based to activity-based measures of
accessibility.''} \emph{Transportation Research Part A: Policy and
Practice}, 40 (2): 163--180.
\url{https://doi.org/10.1016/j.tra.2005.05.002}.

\bibitem[\citeproctext]{ref-donnelly2017}
Donnelly, R. 2017. \emph{{SWIM Version 2.5 Model Development Report}}.
Retrieved from \url{https://github.com/tlumip/model-dev-report/}.

\bibitem[\citeproctext]{ref-ganin2017}
Ganin, A. A., M. Kitsak, D. Marchese, J. M. Keisler, T. Seager, and I.
Linkov. 2017. {``Resilience and efficiency in transportation
networks.''} \emph{Science advances}, 3 (12): e1701079. American
Association for the Advancement of Science.

\bibitem[\citeproctext]{ref-gecchele2019}
Gecchele, G., R. Ceccato, and M. Gastaldi. 2019. {``Road Network
Vulnerability Analysis: Case Study Considering Travel Demand and
Accessibility Changes.''} \emph{Journal of Transportation Engineering,
Part A: Systems}, 145 (7): 05019004.
\url{https://doi.org/10.1061/JTEPBS.0000252}Publisher: American Society
of Civil Engineers.

\bibitem[\citeproctext]{ref-geurs2004}
Geurs, K. T., and B. van Wee. 2004. {``Accessibility evaluation of
land-use and transport strategies: Review and research directions.''}
\emph{Journal of Transport Geography}, 12 (2): 127--140.
\url{https://doi.org/10.1016/j.jtrangeo.2003.10.005}.

\bibitem[\citeproctext]{ref-geurs2010}
Geurs, K., B. Zondag, G. de Jong, and M. de Bok. 2010. {``Accessibility
appraisal of land-use/transport policy strategies: More than just adding
up travel-time savings.''} \emph{Transportation Research Part D:
Transport and Environment}, 15 (7): 382--393.
\url{https://doi.org/10.1016/j.trd.2010.04.006}Specification and
interpretation issues in behavioural models used for environmental
assessment.

\bibitem[\citeproctext]{ref-guze2014}
Guze, S. 2014. {``Graph theory approach to transportation systems design
and optimization.''} \emph{TransNav, the International Journal on Marine
Navigation and Safety of Sea Transportation}, 8 (4): 571--578. Gdynia
Maritime University, Faculty of Navigation.
\url{https://doi.org/10.12716/1001.08.04.12}.

\bibitem[\citeproctext]{ref-hamedi2018}
Hamedi, M., S. Eshragh, M. Franz, and P. M. Sekula. 2018.
\emph{Analyzing impact of i-85 bridge collapse on regional travel in
atlanta}.

\bibitem[\citeproctext]{ref-handy1997}
Handy, S. L., and D. A. Niemeier. 1997. {``Measuring accessibility: An
exploration of issues and alternatives.''} \emph{Environment and
Planning A: Economy and Space}, 29 (7): 1175--1194.
\url{https://doi.org/10.1068/a291175}.

\bibitem[\citeproctext]{ref-helton2003}
Helton, J. C., and F. J. Davis. 2003. {``Latin hypercube sampling and
the propagation of uncertainty in analyses of complex systems.''}
\emph{Reliability Engineering \& System Safety}, 81 (1): 23--69.
\url{https://doi.org/10.1016/S0951-8320(03)00058-9}.

\bibitem[\citeproctext]{ref-ibrahim2011}
Ibrahim, S., R. Ammar, S. Rajasekaran, N. Lownes, Q. Wang, and D.
Sharma. 2011. {``\href{https://doi.org/10.1109/ISCC.2011.5983988}{An
efficient heuristic for estimating transportation network
vulnerability}.''} \emph{2011 IEEE symposium on computers and
communications (ISCC)}, 1092--1098.

\bibitem[\citeproctext]{ref-ip2011}
Ip, W. H., and D. Wang. 2011. {``Resilience and friability of
transportation networks: Evaluation, analysis and optimization.''}
\emph{IEEE Systems Journal}, 5 (2): 189--198.
\url{https://doi.org/10.1109/JSYST.2010.2096670}.

\bibitem[\citeproctext]{ref-jaller2015}
Jaller, M., C. A. G. Calderón, W. F. Yushimito, and I. D. S. Díaz. 2015.
{``An investigation of the effects of critical infrastructure on urban
mobility in the city of medellín.''} \emph{International Journal of
Critical Infrastructures}, 11 (3): 213.
\url{https://doi.org/10.1504/ijcis.2015.072158}.

\bibitem[\citeproctext]{ref-jenelius2012}
Jenelius, E., and L.-G. Mattsson. 2012. {``Road network vulnerability
analysis of area-covering disruptions: A grid-based approach with case
study.''} \emph{Transportation Research Part A: Policy and Practice},
Network vulnerability in large-scale transport networks, 46 (5):
746--760. \url{https://doi.org/10.1016/j.tra.2012.02.003}.

\bibitem[\citeproctext]{ref-kaddoura2018}
Kaddoura, I., and K. Nagel. 2018. {``Using real-world traffic incident
data in transport modeling.''} \emph{Procedia Computer Science}, The 9th
international conference on ambient systems, networks and technologies
({ANT} 2018) / the 8th international conference on sustainable energy
information technology ({SEIT-2018}) / affiliated workshops, 130:
880--885. \url{https://doi.org/10.1016/j.procs.2018.04.084}.

\bibitem[\citeproctext]{ref-koppelman2006}
Koppelman, F., and C. Bhat. 2006. {``A self instructing course in mode
choice modeling: Multinomial and nested logit models.''}

\bibitem[\citeproctext]{ref-masiero2012}
Masiero, L., and R. Maggi. 2012. {``Estimation of indirect cost and
evaluation of protective measures for infrastructure vulnerability: A
case study on the transalpine transport corridor.''} \emph{Transport
Policy}, 20: 13--21. Elsevier.

\bibitem[\citeproctext]{ref-miller2015}
Miller, M., S. Cortes, D. Ory, and J. W. Baker. 2015.
{``\href{https://trid.trb.org/View/1337593}{94th annual
MeetingTransportation research board}.''} Washington, D.C. Number:
15-2366.

\bibitem[\citeproctext]{ref-omer2013}
Omer, M., A. Mostashari, and R. Nilchiani. 2013. {``Assessing resilience
in a regional road-based transportation network.''} \emph{International
Journal of Industrial and Systems Engineering}, 13 (4): 389--408.
\url{https://doi.org/10.1504/ijise.2013.052605}.

\bibitem[\citeproctext]{ref-osei2014}
Osei-Asamoah, A., and N. E. Lownes. 2014. {``Complex network method of
evaluating resilience in surface transportation networks.''}
\emph{Transportation Research Record}, 2467 (1): 120--128.
\url{https://doi.org/10.3141/2467-13}.

\bibitem[\citeproctext]{ref-peeta2010}
Peeta, S., F. Sibel Salman, D. Gunnec, and K. Viswanath. 2010.
{``Pre-disaster investment decisions for strengthening a highway
network.''} \emph{Computers \& Operations Research}, 37 (10):
1708--1719. \url{https://doi.org/10.1016/j.cor.2009.12.006}.

\bibitem[\citeproctext]{ref-salon2021}
Salon, D., M. W. Conway, D. Capasso da Silva, R. S. Chauhan, S.
Derrible, A. (Kouros). Mohammadian, S. Khoeini, N. Parker, L. Mirtich,
A. Shamshiripour, E. Rahimi, and R. M. Pendyala. 2021. {``The potential
stickiness of pandemic-induced behavior changes in the united states.''}
\emph{Proceedings of the National Academy of Sciences}, 118 (27):
e2106499118. \url{https://doi.org/10.1073/pnas.2106499118}Publisher:
Proceedings of the National Academy of Sciences.

\bibitem[\citeproctext]{ref-scott2006}
Scott, D. M., D. C. Novak, L. Aultman-Hall, and F. Guo. 2006. {``Network
robustness index: A new method for identifying critical links and
evaluating the performance of transportation networks.''} \emph{Journal
of Transport Geography}, 14 (3): 215--227.
\url{https://doi.org/10.1016/j.jtrangeo.2005.10.003}.

\bibitem[\citeproctext]{ref-serulle2011}
Serulle, N. U., K. Heaslip, B. Brady, W. C. Louisell, and J. Collura.
2011. {``Resiliency of transportation network of santo domingo,
dominican republic: Case study.''} \emph{Transportation Research
Record}, 2234 (1): 22--30. \url{https://doi.org/10.3141/2234-03}.

\bibitem[\citeproctext]{ref-taylor2008}
Taylor, M. A. P. 2008. {``Critical transport infrastructure in urban
areas: Impacts of traffic incidents assessed using accessibility-based
network vulnerability analysis.''} \emph{Growth and Change}, 39 (4):
593--616. \url{https://doi.org/10.1111/j.1468-2257.2008.00448.x}.

\bibitem[\citeproctext]{ref-taylor2012}
Taylor, M. A. P., and Susilawati. 2012. {``Remoteness and accessibility
in the vulnerability analysis of regional road networks.''}
\emph{Transportation Research Part A: Policy and Practice}, Network
vulnerability in large-scale transport networks, 46 (5): 761--771.
\url{https://doi.org/10.1016/j.tra.2012.02.008}.

\bibitem[\citeproctext]{ref-thill2005}
Thill, J.-C., and M. Kim. 2005. {``Trip making, induced travel demand,
and accessibility.''} \emph{Journal of Geographical Systems}, 7 (2):
229--248. \url{https://doi.org/10.1007/s10109-005-0158-3}.

\bibitem[\citeproctext]{ref-UDOT2020}
UDOT. 2020. {``UDOT asset risk management process.''} Available at
https://drive.google.com/file/d/1lCjChiEnEBqT8gAcaonIhJ8DRacwy0Lt/view.

\bibitem[\citeproctext]{ref-wang2016}
Virginia Department of Transportation. 2016. {``RVTPO regional model.''}
Available at https://github.com/xinwangvdot/rvtpo.

\bibitem[\citeproctext]{ref-vodak2019}
Vodák, R., M. Bíl, T. Svoboda, Z. Křivánková, J. Kubeček, T. Rebok, and
P. Hliněný. 2019. {``A deterministic approach for rapid identification
of the critical links in networks.''} \emph{Plos One}, 14 (7).
\url{https://doi.org/10.1371/journal.pone.0219658}.

\bibitem[\citeproctext]{ref-volker2020}
Volker, J. M. B., A. E. Lee, and S. Handy. 2020. {``Induced Vehicle
Travel in the Environmental Review Process.''} \emph{Transportation
Research Record}, 2674 (7): 468--479.
\url{https://doi.org/10.1177/0361198120923365}Publisher: SAGE
Publications Inc.

\bibitem[\citeproctext]{ref-winkler2016}
Winkler, C. 2016. {``Evaluating transport user benefits: Adjustment of
logsum difference for constrained travel demand models.''}
\emph{Transportation Research Record}, 2564 (1): 118--126.
\url{https://doi.org/10.3141/2564-13}.

\bibitem[\citeproctext]{ref-xie2011}
Xie, F., and D. Levinson. 2011. {``Evaluating the effects of the i-35W
bridge collapse on road-users in the twin cities metropolitan region.''}
\emph{Transportation Planning and Technology}, 34 (7): 691--703.
Routledge. \url{https://doi.org/10.1080/03081060.2011.602850}.

\bibitem[\citeproctext]{ref-xu2015}
Xu, X., A. Chen, S. Jansuwan, K. Heaslip, and C. Yang. 2015. {``Modeling
transportation network redundancy.''} \emph{Transportation research
procedia}, 9: 283--302. Elsevier.

\bibitem[\citeproctext]{ref-zhan2022}
Zhan, S., A. Chang-Richards, K. Elwood, and M. Boston. 2022.
{``\href{https://repo.nzsee.org.nz/xmlui/handle/nzsee/2507}{Post-earthquake
functional recovery: A critical review}.''} Accepted:
2023-02-21T01:18:17Z Publisher: New Zealand Society for Earthquake
Engineering.

\bibitem[\citeproctext]{ref-zhang2016}
Zhang, W., and N. Wang. 2016. {``Resilience-based risk mitigation for
road networks.''} \emph{Structural Safety}, 62: 57--65.
\url{https://doi.org/10.1016/j.strusafe.2016.06.003}.

\bibitem[\citeproctext]{ref-zhang2015}
Zhang, X., E. Miller-Hooks, and K. Denny. 2015. {``Assessing the role of
network topology in transportation network resilience.''} \emph{Journal
of Transport Geography}, 46: 35--45.
\url{https://doi.org/10.1016/j.jtrangeo.2015.05.006}.

\bibitem[\citeproctext]{ref-zhao2002}
Zhao, Y., and K. M. Kockelman. 2002. {``The propagation of uncertainty
through travel demand models: An exploratory analysis.''} \emph{The
Annals of Regional Science}, 36 (1): 145--163.
\url{https://doi.org/10.1007/s001680200072}.

\bibitem[\citeproctext]{ref-zhu2010}
Zhu, S., D. Levinson, H. X. Liu, and K. Harder. 2010a. {``The traffic
and behavioral effects of the i-35W mississippi river bridge
collapse.''} \emph{Transportation Research Part A: Policy and Practice},
44 (10): 771--784. \url{https://doi.org/10.1016/j.tra.2010.07.001}.

\bibitem[\citeproctext]{ref-levinson2010}
Zhu, S., D. Levinson, H. Liu, K. Harder, and A. Dancyzk. 2010b.
{``Traffic flow and road user impacts of the collapse of the i-35W
bridge over the mississippi river.''} Minnesota Department of
Transportation, Research Services Section.

\end{CSLReferences}



\end{document}
