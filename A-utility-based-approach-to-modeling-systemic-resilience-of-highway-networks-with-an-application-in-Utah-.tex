% Options for packages loaded elsewhere
\PassOptionsToPackage{unicode}{hyperref}
\PassOptionsToPackage{hyphens}{url}
\PassOptionsToPackage{dvipsnames,svgnames,x11names}{xcolor}
%

\documentclass[
  letterpaper,
]{trb}

\usepackage{fullpage}
\usepackage[pagewise]{lineno}
\linenumbers
\usepackage{newtxtext}


\makeatletter
\newcounter{wordcounter}
\setcounter{wordcounter}{5706}

\newcounter{tablecounter}
\setcounter{tablecounter}{4}

\newcounter{totalwordcounter}
\newcommand{\totalwordcount}{%
  \setcounter{totalwordcounter}{5706}
  \addtocounter{totalwordcounter}{\numexpr250*4}
  \number\value{totalwordcounter}% Output the number
	\renewcommand{\totalwordcount}{\number\value{totalwordcounter}}
}
\makeatother

\usepackage{amsmath,amssymb}
\usepackage{iftex}
\ifPDFTeX
  \usepackage[T1]{fontenc}
  \usepackage[utf8]{inputenc}
  \usepackage{textcomp} % provide euro and other symbols
\else % if luatex or xetex
  \usepackage{unicode-math}
  \defaultfontfeatures{Scale=MatchLowercase}
  \defaultfontfeatures[\rmfamily]{Ligatures=TeX,Scale=1}
\fi
\usepackage{lmodern}
\ifPDFTeX\else  
    % xetex/luatex font selection
\fi
% Use upquote if available, for straight quotes in verbatim environments
\IfFileExists{upquote.sty}{\usepackage{upquote}}{}
\IfFileExists{microtype.sty}{% use microtype if available
  \usepackage[]{microtype}
  \UseMicrotypeSet[protrusion]{basicmath} % disable protrusion for tt fonts
}{}
\makeatletter
\@ifundefined{KOMAClassName}{% if non-KOMA class
  \IfFileExists{parskip.sty}{%
    \usepackage{parskip}
  }{% else
    \setlength{\parindent}{0pt}
    \setlength{\parskip}{6pt plus 2pt minus 1pt}}
}{% if KOMA class
  \KOMAoptions{parskip=half}}
\makeatother
\usepackage{xcolor}
\setlength{\emergencystretch}{3em} % prevent overfull lines
\setcounter{secnumdepth}{5}
% Make \paragraph and \subparagraph free-standing
\ifx\paragraph\undefined\else
  \let\oldparagraph\paragraph
  \renewcommand{\paragraph}[1]{\oldparagraph{#1}\mbox{}}
\fi
\ifx\subparagraph\undefined\else
  \let\oldsubparagraph\subparagraph
  \renewcommand{\subparagraph}[1]{\oldsubparagraph{#1}\mbox{}}
\fi


\providecommand{\tightlist}{%
  \setlength{\itemsep}{0pt}\setlength{\parskip}{0pt}}\usepackage{longtable,booktabs,array}
\usepackage{calc} % for calculating minipage widths
% Correct order of tables after \paragraph or \subparagraph
\usepackage{etoolbox}
\makeatletter
\patchcmd\longtable{\par}{\if@noskipsec\mbox{}\fi\par}{}{}
\makeatother
% Allow footnotes in longtable head/foot
\IfFileExists{footnotehyper.sty}{\usepackage{footnotehyper}}{\usepackage{footnote}}
\makesavenoteenv{longtable}
\usepackage{graphicx}
\makeatletter
\def\maxwidth{\ifdim\Gin@nat@width>\linewidth\linewidth\else\Gin@nat@width\fi}
\def\maxheight{\ifdim\Gin@nat@height>\textheight\textheight\else\Gin@nat@height\fi}
\makeatother
% Scale images if necessary, so that they will not overflow the page
% margins by default, and it is still possible to overwrite the defaults
% using explicit options in \includegraphics[width, height, ...]{}
\setkeys{Gin}{width=\maxwidth,height=\maxheight,keepaspectratio}
% Set default figure placement to htbp
\makeatletter
\def\fps@figure{htbp}
\makeatother
\newlength{\cslhangindent}
\setlength{\cslhangindent}{1.5em}
\newlength{\csllabelwidth}
\setlength{\csllabelwidth}{3em}
\newlength{\cslentryspacingunit} % times entry-spacing
\setlength{\cslentryspacingunit}{\parskip}
\newenvironment{CSLReferences}[2] % #1 hanging-ident, #2 entry spacing
 {% don't indent paragraphs
  \setlength{\parindent}{0pt}
  % turn on hanging indent if param 1 is 1
  \ifodd #1
  \let\oldpar\par
  \def\par{\hangindent=\cslhangindent\oldpar}
  \fi
  % set entry spacing
  \setlength{\parskip}{#2\cslentryspacingunit}
 }%
 {}
\usepackage{calc}
\newcommand{\CSLBlock}[1]{#1\hfill\break}
\newcommand{\CSLLeftMargin}[1]{\parbox[t]{\csllabelwidth}{#1}}
\newcommand{\CSLRightInline}[1]{\parbox[t]{\linewidth - \csllabelwidth}{#1}\break}
\newcommand{\CSLIndent}[1]{\hspace{\cslhangindent}#1}

\usepackage{booktabs}
\usepackage{longtable}
\usepackage{array}
\usepackage{multirow}
\usepackage{wrapfig}
\usepackage{float}
\usepackage{colortbl}
\usepackage{pdflscape}
\usepackage{tabu}
\usepackage{threeparttable}
\usepackage{threeparttablex}
\usepackage[normalem]{ulem}
\usepackage{makecell}
\usepackage{xcolor}
\usepackage{orcidlink}
\definecolor{mypink}{RGB}{219, 48, 122}
\makeatletter
\makeatother
\makeatletter
\@ifpackageloaded{bookmark}{}{\usepackage{bookmark}}
\makeatother
\makeatletter
\@ifpackageloaded{caption}{}{\usepackage{caption}}
\AtBeginDocument{%
\ifdefined\contentsname
  \renewcommand*\contentsname{Table of contents}
\else
  \newcommand\contentsname{Table of contents}
\fi
\ifdefined\listfigurename
  \renewcommand*\listfigurename{List of Figures}
\else
  \newcommand\listfigurename{List of Figures}
\fi
\ifdefined\listtablename
  \renewcommand*\listtablename{List of Tables}
\else
  \newcommand\listtablename{List of Tables}
\fi
\ifdefined\figurename
  \renewcommand*\figurename{Figure}
\else
  \newcommand\figurename{Figure}
\fi
\ifdefined\tablename
  \renewcommand*\tablename{Table}
\else
  \newcommand\tablename{Table}
\fi
}
\@ifpackageloaded{float}{}{\usepackage{float}}
\floatstyle{ruled}
\@ifundefined{c@chapter}{\newfloat{codelisting}{h}{lop}}{\newfloat{codelisting}{h}{lop}[chapter]}
\floatname{codelisting}{Listing}
\newcommand*\listoflistings{\listof{codelisting}{List of Listings}}
\makeatother
\makeatletter
\@ifpackageloaded{caption}{}{\usepackage{caption}}
\@ifpackageloaded{subcaption}{}{\usepackage{subcaption}}
\makeatother
\makeatletter
\@ifpackageloaded{tcolorbox}{}{\usepackage[skins,breakable]{tcolorbox}}
\makeatother
\makeatletter
\@ifundefined{shadecolor}{\definecolor{shadecolor}{rgb}{.97, .97, .97}}
\makeatother
\makeatletter
\makeatother
\makeatletter
\makeatother
\ifLuaTeX
  \usepackage{selnolig}  % disable illegal ligatures
\fi
\IfFileExists{bookmark.sty}{\usepackage{bookmark}}{\usepackage{hyperref}}
\IfFileExists{xurl.sty}{\usepackage{xurl}}{} % add URL line breaks if available
\urlstyle{same} % disable monospaced font for URLs
\hypersetup{
  pdftitle={A utility-based approach to modeling systemic resilience of highway networks with an application in Utah},
  pdfauthor={Gregory S. Macfarlane; Max Barnes; Natalie Mae Gray},
  colorlinks=true,
  linkcolor={blue},
  filecolor={Maroon},
  citecolor={Blue},
  urlcolor={red},
  pdfcreator={LaTeX via pandoc}}


\title{A utility-based approach to modeling systemic resilience of
highway networks with an application in Utah}
\author{
Gregory S. Macfarlane\\Brigham Young University\\Civil and Construction
Engineering\\\href{mailto:gregmacfarlane@byu.edu}{gregmacfarlane@byu.edu}\\ORCID: 0000-0003-3999-7584\\ \\ 
Max
Barnes\\Kimley-Horn\\\\\href{mailto:maxbarnes@kha.com}{maxbarnes@kha.com}\\\\ \\ 
Natalie Mae Gray\\Brigham Young
University\\\\\href{mailto:nmgray@byu.edu}{nmgray@byu.edu}\\}
\date{2023-08-01}
\begin{document}
\maketitle
\newpage
\begin{abstract}
The resilience of transportation networks is an important consideration
in management and planning, but practical techniques to identify
systemically critical links are limited. Further, current practical
techniques ignore that when transportation networks are damaged or
degraded, people potentially change destinations and modes as well as
travel routes. In this research, we develop a model to examine network
highway resilience based on changes to mode and destination choice
logsums, and apply this model to 41 scenarios representing the loss of
links on the statewide highway network in Utah. The results of the
analysis suggest a fundamentally different prioritization scheme than
would be identified solely through a methodology based on increased
travel times.
\end{abstract}
\newpage
\ifdefined\Shaded\renewenvironment{Shaded}{\begin{tcolorbox}[interior hidden, breakable, borderline west={3pt}{0pt}{shadecolor}, boxrule=0pt, sharp corners, enhanced, frame hidden]}{\end{tcolorbox}}\fi

\bookmarksetup{startatroot}

\hypertarget{intro}{%
\section{Introduction}\label{intro}}

Systemic resiliency is an important consideration for transportation
agencies, though specific definitions of ``resiliency'' might vary under
different contexts. Some agencies and researchers see resiliency as
facility-level design and engineering that hardens the system against
failure (\emph{1}, \emph{2}); others as an ability for maintenance staff
to rapidly restore service following catastrophe (\emph{3}); and others
as an ability for a system to continue operating in degraded state
(\emph{4}, \emph{5}). Regardless of the definition used, assessing the
resiliency of a transportation network --- and addressing any potential
shortfalls --- requires a method to identify which links or facilities
are most critical to the smooth operation of the network.

Identifying which links in a transportation network are most critical,
however, is not trivial. State of the practice techniques typically rely
on traffic volumes represented in terms of vehicles, trips, or freight
value (e.g. \emph{6}). But these methods ignore the fact that some
networks have alternate routes readily available, have multiple well
developed modes of transportation, or have redundant destination
locations. When a network link does break, routes, mode choice, and
destination choice often change as well. These responses have been
observed in real-world crisis events like the I-35 W bridge collapse in
Minneapolis and the I-85 bridge fire and collapse in Atlanta
(\emph{7}--\emph{10}).

In this research, we apply a destination choice accessibility model to
identify critical highway links in a statewide highway context. This
model is designed to capture the utility lost to individuals operating
on a degraded network who may choose a longer route, a different travel
mode, or a different destination. We then apply the model to a highway
network representing the state of Utah. The model structure permits a
more nuanced evaluation of link criticality than using traffic volume
alone or in conjunction with an increased travel time measure.

The paper proceeds as follows. First, a
\protect\hyperlink{litreview}{literature review} discusses previous
attempts to analyze the resiliency of transportation networks. A
\protect\hyperlink{methodology}{methodology} section presents the model
developed for this research and describes the implementation process in
Utah. A \protect\hyperlink{results}{results} section describes the model
output in detail for one scenario and then compares and ranks the model
outputs for an array of several dozen independent scenarios. The paper
\protect\hyperlink{conclusion}{concludes} with a discussion of
limitations and associated avenues for future research.

\bookmarksetup{startatroot}

\hypertarget{litreview}{%
\section{Literature}\label{litreview}}

In a groundbreaking theoretical article, Berdica (\emph{4}) attempted to
identify, define and conceptualize network ``vulnerability'' --- the
complement of resiliency --- by envisioning analyses conducted with
several vulnerability performance measures including travel time, delay,
congestion, serviceability, and accessibility. She then defined
reliability as the level of reduced accessibility due to unfavorable
operating conditions on the network. In particular, the author
identifies a need for further research toward developing a framework
capable of investigating reliability of transportation networks.

In this section we examine several attempts by numerous researchers to
do precisely this using various measures of network performance. It is
helpful to categorize the existing literature into three groups
(summarized in Table~\ref{tbl-authortable}) based on the overall
technique applied in the study. These groups include:

\begin{itemize}
\item
  \emph{Network connectivity}: How does damage to a network diminish the
  connectivity between network nodes?
\item
  \emph{Travel time analysis}: How much do shortest path travel times
  between origins and destinations increase on a damaged network?
\item
  \emph{Accessibility analysis} How easily can the population using the
  damaged network complete their daily activities?
\end{itemize}

\hypertarget{tbl-authortable}{}
\begin{table}
\caption{\label{tbl-authortable}Attempts to Evaluate Systemic Resiliency }\tabularnewline

\centering
\begin{tabular}[t]{rll}
\toprule
Year & Author & Performance Metric\\
\midrule
2004 & Geurs and van Wee & Accessibility (isochrone, gravity, logsum)\\
2007 & Abdel-Rahim et al. & Network Connectivity\\
2008 & Taylor, M & Accessibility (logsum)\\
2010 & Peeta et al. & Travel time and cost\\
2010 & Geurs et al. & Accessibility (logsum)\\
\addlinespace
2010 & Levinson and Zhu & Travel time and cost\\
2010 & Zhu et al. & Travel time and cost\\
2011 & Agarwal et al. & Network connectivity\\
2011 & Ip and Wang & Network connectivity\\
2011 & Serulle et al. & Travel time and cost\\
\addlinespace
2011 & Ibrahim, S & Travel time and cost\\
2011 & Xie and Levinson & Accessibility (isochrone)\\
2013 & Omer et al. & Travel time and cost\\
2014 & Osei-Asamoah and Lownes & Network connectivity\\
2015 & Zhang et al. & Network connectivity\\
\addlinespace
2015 & Guze & Network connectivity\\
2015 & Jaller et al. & Travel time and cost\\
2015 & Xu et al. & Network connectivity\\
2016 & Winkler, C. & Accessibility (gravity)\\
2017 & Ganin et al. & Accessibility (gravity)\\
\addlinespace
2019 & Vodak et al. & Network connectivity\\
2019 & Hackl and Adey & Network connectivity\\
\bottomrule
\end{tabular}
\end{table}

The purpose of transportation networks is to connect locations to each
other; presumably damage to a network would diminish the network's
\emph{connectivity}, or the number of paths between node pairs. It may
even leave nodes or groups of nodes completely isolated. In studies
using connectivity as the primary performance measure, researchers
typically apply methods and concepts from graph theory. These measures
may include elementary measures such as the isolation of nodes in a
network (\emph{11}). More advanced measures have included heirarchical
clustering of node paths (\emph{12}, \emph{13}), a count of independent
paths (\emph{14}), the reduction of total network capacity (\emph{5}),
and special applications of the knapsack and traveling salesman problems
(\emph{15}, \emph{16}). Though useful from a theoretical perspective,
many of these authors reported that their approaches tend to break down
to some degree on large, real-world networks where the number of nodes
and links numbers in the tens of thousands, and the degree of
connectivity between any arbitrary node pair is high. They also do not
typically account for how network users may react to the new topology or
capacity constraints of the degraded network.

Highway system network failures --- in most imaginable cases --- degrade
the shortest or least cost path, but typically do not eliminate all
paths. The degree to which travel time increases when a particular link
is damaged, however, could provide an estimate of the criticality of
that link or node. This general method has been used to evaluate
potential choke points in various networks (\emph{17}--\emph{19}) as
well as the allocation of emergency resources (\emph{2}). Though many
applications only consider the increase in travel time, some authors
consider how the users of the network will respond to the decreased
capacity (\emph{20}--\emph{22}), and others attempt to model a shift in
departure time or mode (\emph{23}).

A primary limitation with increased travel time methodologies is that
they ignore other possible ways a population might adapt its travel to a
damaged network. Aside from shifting routes and modes, people may choose
other destinations and it is possible that some previously planned trips
might be canceled entirely. Travel time-based methods do not account for
the costs of these changes in plans. But accessibility methods --- in
particular the accessibility calculations embedded within many existing
regional transport models ---- provide a framework for evaluating these
costs (\emph{24}, \emph{25}).

Accessibility is an abstract concept with multiple methods of
quantification (\emph{26}). Perhaps the most popular method is the
cumulative opportunities measure: e.g., the number of jobs within a
specified travel time threshold. This is the method employed by Xie and
Levinson (\emph{9}) in an analysis of the impact of the I-35 W bridge
collapse in Minneapolis. But cumulative opportunities measures require
the analyst to assert a travel time threshold, a mode, and an
opportunity of interest. Some of these assumptions can be relaxed with a
gravity-style access model, but the logsum of a destination choice model
has several benefits as an accessibility term including its grounding in
choice framework, ability to weigh multiple attributes of an
alternative, and include travel impedances by all modes (\emph{27}).
These measures can even be weighted by a cost coefficient to translate
the lost utility in monetary terms (\emph{25}, \emph{28}).

Logsum-derived accessibility meaures have been used before to evaluate
network resiliency (\emph{29}--\emph{31}). Where these previous efforts
have been somewhat limited is in their theoretical nature. Though
previous researchers have shown that logsum-derived accessibility
measures are feasible and informative, their use in actual resilience
analysis efforts by state departments of transportation and other
relevant agencies appears limited.

\bookmarksetup{startatroot}

\hypertarget{methodology}{%
\section{Methodology}\label{methodology}}

In this section we describe a model framework designed to evaluate
resilience using a logit-based choice metric. This framework is heavily
based on tools and methods in existing statewide travel models, with a
few necessary extensions. We then describe the implementation of this
model framework to a prioritization exercise on the Utah Statewide
Travel Model (USTM).

\hypertarget{model-design}{%
\subsection{Model Design}\label{model-design}}

The overall model framework is presented in Figure~\ref{fig-framework},
and is designed to capture the utility-based accessibility for a
particular origin zones \(i\) and trip purpose \(m\). The model begins
with a travel time skim procedure, to determine the congested travel
time from zone \(i\) to zone \(j\) by auto as well as the shortest
network distance for non motorized modes. The transit travel time skim
is fixed, assuming that transit infrastructure would not be affected by
changes to the highway network. Throughout this section, lower-cased
index variables \(k\) belong to a set of all indices described by the
corresponding capital letter \(K\).

\begin{figure}

{\centering \includegraphics[width=1\textwidth,height=\textheight]{03-methods_files/figure-pdf/fig-framework-1.pdf}

}

\caption{\label{fig-framework}Model framework with feedback cycle. Blue
boxes are calculated after the second feedback loop.}

\end{figure}

With the travel time \(t_{ijk}\) for all modes \(k \in K\), the model
computes mode choice utility values. The multinomial logit mode choice
model describes the probability of a person at origin \(i\) choosing
mode \(k\) for a trip to destination \(j\):
\begin{equation}\protect\hypertarget{eq-mcp}{}{
\mathcal{P}_{ijm}(k) = \frac{\exp(f(\beta_{m}, t_{ijk}))}{\sum_{K}\exp(f(\beta_{m}, t_{ijk}))}
}\label{eq-mcp}\end{equation} The log of the denominator of the this
equation is called the mode choice logsum, \(MCLS_{ijm}\) and is a
measure of the travel cost by all modes, weighted by utility parameters
\(\beta_m\) that may vary by trip purpose.

The \(MCLS\) is then used as a travel impedance term in the multinomial
logit destination choice model, where the probability of a person at
origin \(i\) choosing destination \(j \in J\) is
\begin{equation}\protect\hypertarget{eq-dcp}{}{
\mathcal{P}_{im}(j) = \frac{\exp(f(\gamma_{m}, MCLS_{ijm}, A_j))}{\sum_{J}\exp(f(\gamma_{m}, MCLS_{ijm}, A_j))}
}\label{eq-dcp}\end{equation} where \(A_j\) is the attractiveness ---
represented in terms of socioeconomic activity --- of zone \(j\). As
with mode choice, the log of the denominator of this model is the
destination choice logsum, \(DCLS_{im}\). This quantity represents the
value access to all destinations by all modes of travel, and varies by
trip purpose.

The \(DCLS_{im}\) measure is relative, but can be compared across
scenarios. The difference between the measures of two scenarios
\begin{equation}\protect\hypertarget{eq-deltas}{}{
\Delta_{im} = DCLS_{im}^{\mathrm{Base}} - DCLS_{im}^{\mathrm{Scenario}}
}\label{eq-deltas}\end{equation} provides an estimate of the
accessibility lost when \(t_{ij\mathrm{drive}}\) changes due to a
damaged highway link. This accessibility change is \emph{per trip},
meaning that the total lost accessibility is \(P_{im} * \Delta_{im}\)
where \(P\) is the number of trip productions at zone \(i\) for purpose
\(m\). This measure is given in units of dimensionless utility, but the
mode choice cost coefficient \(\beta_{\mathrm{cost}}\) provides a
conversion factor between utility and cost. The total financial cost of
a damaged link for the entire region for all trip purposes is
\begin{equation}\protect\hypertarget{eq-totalcost}{}{
\mathrm{Cost} = \sum_{I}\sum_{M} -1 / \beta_{\mathrm{cost},m} * P_{im} \Delta_{im}
}\label{eq-totalcost}\end{equation}

For comparison to a simpler resiliency method that only includes the
increased travel time between origins and destinations, we compute the
change in congested travel time between \(\delta t_{ij}\) and multiply
the number of trips by this change and a value of time coefficient
derived from the cost and vehicle time coefficients of the mode choice
model, \begin{equation}\protect\hypertarget{eq-ttmethod}{}{
\mathrm{Cost}' =  \sum_I \sum_J \sum_M \frac{\beta_{\mathrm{time}, m} }{\beta_{\mathrm{cost}, m}} T_{ijm} \delta t_{ijm}
}\label{eq-ttmethod}\end{equation}

\hypertarget{model-implementation-in-utah}{%
\subsection{Model Implementation in
Utah}\label{model-implementation-in-utah}}

The Utah Department of Transportation (UDOT) manages an extensive
highway network consisting of interstate freeways (I-15, I-80, I-70, and
I-84), intraurban expressways along the Wasatch Front, and rural
highways throughout the state. The rugged mountain and canyon topography
throughout the state places severe constraints on possible redundant
paths in the highway network. A landslide or rock fall in any single
canyon may isolate a community or force a redirection of traffic that
could be several hours longer than the preferred route; understanding
which of these many possible choke points is most critical is a key and
ongoing objective of the agency.

Several data elements for the model described above were obtained from
the Utah Statewide Travel Model (USTM). USTM is a trip-based statewide
model that is focused exclusively on long-distance and rural trips:
intraurban trips within existing Metropolitan Planning Organization
(MPO) model regions are pre-loaded onto the USTM highway network. This
means that USTM as currently constituted can be used for infrastructure
planning purposes, but would be inadequate to evaluate the systemic
resiliency of the highway network given the disparate methodologies of
the MPO models. USTM can, however, provide the following data elements

\begin{enumerate}
\def\labelenumi{\arabic{enumi}.}
\tightlist
\item
  \emph{Highway Network}: including free flow and congested travel
  speeds, link length, link capacity estimates, etc.
\item
  \emph{Zonal Productions} \(P_{im}\): available for all zones by
  purpose, including those in the MPO region areas.
\item
  \emph{Zonal Socioeconomic Data}: the destination choice model
  described in Equation @ref(eq:dcp) calculates attractions \(A_{jm}\)
  from the USTM zonal socioeconomic data based on the utility
  coefficients in Table~\ref{tbl-coeffs}.
\item
  \emph{Calibration Targets}: USTM base scenario estimates of mode split
  and trip length were used to calibrate the utility coefficients as
  described below.
\end{enumerate}

Among MPO models in Utah, only the model jointly operated by the Wasatch
Front Regional Council (WFRC, Salt Lake area MPO) and the Mountainland
Association of Governments (MAG, Provo area MPO) model includes a
substantive transit forecasting component. The transit travel time skim
from the WFRC / MAG model was used for the mode choice model in
Equation~\ref{eq-mcp}; the zonal travel time between the smaller WFRC /
MAG model zones was averaged to the larger USTM zones, and the minimum
time among the several modes available (commuter rail, light rail, bus
rapid transit, local bus) was taken as the travel time for a single
transit mode in this implementation.

\hypertarget{tbl-coeffs}{}
\begin{table}
\caption{\label{tbl-coeffs}Choice Model Coefficients }\tabularnewline

\centering
\begin{tabular}[t]{llrrr}
\toprule
 & Variable & HBW & HBO & NHB\\
\midrule
\addlinespace[0.3em]
\multicolumn{5}{l}{\textbf{Destination Choice}}\\
\hspace{1em} & Households & 0.0000 & 1.0187 & 0.2077\\

\hspace{1em} & Office Employment & 0.4568 & 0.4032 & 0.2816\\

\hspace{1em} & Other Employment & 1.6827 & 0.4032 & 0.2816\\

\hspace{1em} & Retail Employment & 0.6087 & 3.8138 & 5.1186\\

\hspace{1em} & Distance & -0.0801 & -0.1728 & -0.1157\\

\hspace{1em} & Distance\textasciicircum{}2 & 0.0026 & 0.0034 & 0.0035\\

\hspace{1em} & Distance\textasciicircum{}3 & 0.0000 & 0.0000 & 0.0000\\
\cmidrule{1-5}
\addlinespace[0.3em]
\multicolumn{5}{l}{\textbf{Mode Choice}}\\
\hspace{1em} & Shared & -1.1703 & 0.0164 & -0.0336\\

\hspace{1em} & Transit & -0.3903 & -1.9811 & -2.2714\\

\hspace{1em} & Non-Motorized & -1.2258 & -0.3834 & -0.8655\\

\hspace{1em} & Travel Time [minutes] & -0.0450 & -0.0350 & -0.0400\\

\hspace{1em} & Travel Cost [dollars] & -0.0016 & -0.0016 & -0.0016\\

\hspace{1em} & Walk Distance (less than 1 mile) [miles] & -0.0900 & -0.0700 & -0.0800\\

\hspace{1em} & Walk Distance (1 mile or more) [miles] & -0.1350 & -0.1050 & -0.1200\\
\bottomrule
\end{tabular}
\end{table}

The utility coefficients for the destination and mode choice models are
presented in Table~\ref{tbl-coeffs}. The mode choice coefficients were
adapted from USTM and supplemented with coefficients from the Roanoke
(Virginia) Valley Transportation Planning Organization (RVTPO) travel
model. This model was selected for its simplicity and analogous data
elements to the proposed model. The alternative-specific constants were
calibrated to regional mode choice targets developed from the 2015 Utah
Household Travel Survey (UHTS) using methods described by (\emph{32}).

The destination choice utility equation consists of three parts: a size
term, a travel impedance term, and a calibration polynomial.
Coefficients for the size term and travel impedance terms were adapted
from the Oregon Statewide Integrated Model for all purposes except HBW.
Instead, these coefficients were adapted from the RVTPO model. The
distance polynomial coefficients were calibrated to targets developed
from the 2012 Utah household travel survey.

\hypertarget{vulnerable-link-identification}{%
\subsubsection{Vulnerable Link
Identification}\label{vulnerable-link-identification}}

To develop evaluation scenarios on which to apply the model, we used
information contained in the UDOT Risk Priority Analysis online map
(\emph{33}). This map considers the probability of various events that
could impact road performance including rock falls, avalanches,
landslides, and other similar occurrences. Using this tool, combined
with information gathered from the research team and UDOT officials, 41
locations of interest were identified for analysis. Each link was
identified due to its location in relation to population centers, remote
geographic location, and proximity to other highway facilities, or
because the link was known to be at risk due to geologic or geographic
features, or because it was a suspected choke point in the network.

\bookmarksetup{startatroot}

\hypertarget{sec-results}{%
\section{Results}\label{sec-results}}

In this section we apply the model to scenarios where critical highway
links are removed from the model network. This includes first a detailed
analysis of a single scenario, where I-80 between Salt Lake and Tooele
Counties is severed. We compare the model output to an alternative
method that measures only the change in travel time and does not allow
for mode or destination choice. The model was then applied to 40
additional link closure scenarios throughout the state.

\hypertarget{detailed-scenario-analysis}{%
\subsection{Detailed Scenario
Analysis}\label{detailed-scenario-analysis}}

To illustrate how the logsum-based model framework captures the costs of
losing a link, we conducted a scenario where I-80 between Tooele and
Salt Lake Counties west of the Salt Lake City International Airport
becomes unavailable. Tooele is a largely rural county with increasing
numbers of residents who commute to jobs in the Salt Lake Valley. I-80
is the only realistic path between the two counties, as alternate routes
involve mountain canyons and many additional miles of travel.

The costs obtained by the logsum and travel time based methods for this
scenario are shown in Table~\ref{tbl-tooeletable}. In both analyses, we
separate the costs spatially, though the definitions of the two are
slightly different. In the logsum-based method, ``Inside Toole'' are
costs associated with decreased accessibility for trips produced in
Tooele County. The increased costs in this case capture the loss in
utility by measuring increased multi-model travel impedances to
destinations with high attractiveness. In the travel-time method, by
contrast, the ``Inside Tooele'' costs are those for trips with an origin
in Tooele County and a destination in Salt Lake County, and are the
increase in travel time multiplied by a value of time and the number of
vehicles making the trip. The difference in definition is required by
the difference in framework construction.

In general, the logsum-based method arrives at cost estimates that are
less than half of the comparable estimates of the travel time-based
method. This is not unexpected, as the travel time-based method assumes
that all the preexisting trips would still occur, but on a longer path.
The logsum-based method on the other hand attempts to capture the fact
that when a path changes, people may shift their destination or their
mode of travel. To be clear, the framework we have developed for this
exercise does not explicitly model the selection of a new alternative
destination; rather, we calculate instead the value of a destination
choice set before and after the link is removed. But the implication is
that the availability of alternative destinations still provide some
benefit to the choice maker, a proposition that the travel time method
ignores.

\hypertarget{tbl-tooeletable}{}
\begin{table}
\caption{\label{tbl-tooeletable}Comparison of Logsum-based and Time-based costs for I-80 at Tooele }\tabularnewline

\centering
\resizebox{\linewidth}{!}{
\begin{tabular}[t]{lrrrrrr}
\toprule
\multicolumn{1}{c}{ } & \multicolumn{3}{c}{Utility Logsum} & \multicolumn{3}{c}{Travel Time} \\
\cmidrule(l{3pt}r{3pt}){2-4} \cmidrule(l{3pt}r{3pt}){5-7}
Purpose & Other Counties & Inside Tooele & Total & Other Counties & Inside Tooele & Total\\
\midrule
\addlinespace[0.3em]
\multicolumn{7}{l}{\textbf{Passenger}}\\
\hspace{1em}HBO & \$2,637 & \$28,290 & \$30,927 & \$8,595 & \$75,862 & \$84,457\\
\hspace{1em}HBW & \$4,039 & \$113,660 & \$117,699 & \$7,868 & \$234,009 & \$241,877\\
\hspace{1em}NHB & \$2,141 & \$44,911 & \$47,051 & \$5,875 & \$100,245 & \$106,120\\
\addlinespace[0.3em]
\multicolumn{7}{l}{\textbf{External / Freight}}\\
\hspace{1em}Internal Freight &  &  &  & \$24,617 & \$26,083 & \$50,700\\
\hspace{1em}Inbound / Outbound Freight &  &  &  & \$59,758 & \$1,190 & \$60,948\\
\hspace{1em}Recreation &  &  &  & \$176 & \$190 & \$366\\
\hspace{1em}Through Freight &  &  &  & \$251,508 &  & \$251,508\\
\hspace{1em}Through Passenger &  &  &  & \$56,103 &  & \$56,103\\
Comparable Total & \$8,817 & \$186,860 & \$195,677 & \$22,338 & \$410,116 & \$432,454\\
Total & \$8,817 & \$186,860 & \$195,677 & \$414,499 & \$437,580 & \$852,079\\
\bottomrule
\end{tabular}}
\end{table}

Another key element of Table~\ref{tbl-tooeletable} is that the largest
single element of costs in the scenario is associated with through
freight, as well as contributions from other purposes for which the
logsum model developed in this study did not include a corresponding
methodology. This was due to data limitations and the modeling approach
of the existing USTM, but it is obvious that a choice-based framework
for examining the costs of through and inbound / outbound traffic is an
important limitation in this scenario and potentially many others.

\hypertarget{prioritization-scenario-results}{%
\subsection{Prioritization Scenario
Results}\label{prioritization-scenario-results}}

We now apply the model to compare 40 additional scenarios where
individual highway facilities are removed from the model highway
network. Table~\ref{tbl-traveltimerank} presents the logsum and travel
time results for all 41 scenarios, ordered by decreasing logsum costs.
In other words, I-15 at the boundary between Utah and Salt Lake Counties
is the most ``vulnerable'' or ``critical'' link analyzed in this
exercise. Were this link to be cut, the people of Utah would have the
highest costs per day in lost destination and travel access of any other
link. Each of the highest-ranking roads in this analysis is an
interstate facility in northern Utah, which is more heavily populated
than the remote areas in the south. A map showing the locations of these
scenarios is given in Figure~\ref{fig-linksmap}. The concentration the
highest cost links in the Salt Lake City metropolitan area is obvious,
though the links with the highest cost are not \emph{in} Salt Lake City
where multiple

\begin{figure}

{\centering \includegraphics{04-results_files/figure-pdf/fig-linksmap-1.pdf}

}

\caption{\label{fig-linksmap}Total cost of link closure for each
scenario by the logsum method.}

\end{figure}

Perhaps strangely, many scenarios in the analysis show a \emph{benefit}
from loss of the link. Investigating these scenarios showed that for
many paths, the shortest automobile travel time in the complete network
is \emph{not} the shortest path by distance. When the shortest time path
is disrupted, the new shortest time path may be only a few minutes
longer by time but dozens of miles shorter (on a slower road). Because
the automobile operating costs are calculated per mile and not per
minute, this means that the new path actually produces a benefit. This
challenge is exacerbated by the apparent agency practice of placing
artificial time penalties on the network links in some canyons during
calibration. This exercise reveals one reason why such a practice should
be discouraged, and also highlights the importance of using consistent
functions for impedance calculation at all stages of the model.

\hypertarget{tbl-traveltimerank}{}
\begin{table}
\caption{\label{tbl-traveltimerank}Scenario Results of Both Methodologies }\tabularnewline

\centering
\resizebox{\linewidth}{!}{
\begin{tabular}[t]{crrcrr}
\toprule
\multicolumn{2}{c}{ } & \multicolumn{1}{c}{Logsum} & \multicolumn{3}{c}{Travel Time} \\
\cmidrule(l{3pt}r{3pt}){3-3} \cmidrule(l{3pt}r{3pt}){4-6}
Route & Location & HBW, HBO, NHB & HBW, HBO, NHB & Freight, External, etc. & Total\\
\midrule
I-15 & Utah / Salt Lake county line & \$587,126 & \$827,014 & \$387,138 & \$1,214,152\\
I-80 & Salt Lake / Tooele county line & \$195,677 & \$432,454 & \$419,625 & \$852,079\\
I-84 & Weber Canyon & \$133,705 & \$100,415 & \$87,561 & \$187,976\\
I-80 & Parley's Canyon & \$96,614 & \$77,320 & \$164,493 & \$241,813\\
I-15 & Orem & \$68,707 & \$137,034 & \$105,174 & \$242,208\\
\addlinespace
I-215 & Taylorsville & \$51,327 & \$79,619 & \$2,750 & \$82,370\\
US-91 & Box Elder Canyon & \$39,676 & \$55,515 & \$106,704 & \$162,219\\
SR-189 & Provo Canyon & \$39,088 & \$43,095 & \$13,098 & \$56,193\\
I-15 & SLC  2100 S & \$32,508 & \$98,707 & \$43,222 & \$141,929\\
I-15 & Bountiful & \$28,787 & \$56,806 & \$53,575 & \$110,381\\
\addlinespace
I-15 & Utah / Juab county line & \$28,531 & \$49,815 & \$578,167 & \$627,982\\
Bangerter & West Valley City & \$27,150 & \$27,009 & \$503 & \$27,512\\
SR-18 & Snow Canyon & \$21,857 & \$12,131 & \$199 & \$12,330\\
I-15 & North of Zion & \$13,164 & \$14,793 & \$646,271 & \$661,064\\
I-15 & North of Cove Fort & \$6,313 & \$1,378 & \$393,559 & \$394,937\\
\addlinespace
Timp Highway & American Fork Canyon & \$1,015 & \$2,566 & \$71 & \$2,637\\
Legacy Parkway & West Bountiful & \$464 & \$242 & \$42 & \$284\\
UT-35 & Francis & -\$4,311 & \$2,846 & \$219 & \$3,064\\
SR-14 & Cedar Canyon & -\$5,407 & \$479 & \$136 & \$615\\
US-89 & Logan Canyon & -\$5,515 & \$1,245 & \$5,382 & \$6,626\\
\addlinespace
SR-199 & Rush Valley & -\$5,688 & \$335 & \$77 & \$413\\
SR-62 & Kingston & -\$5,900 & \$779 & \$72 & \$850\\
US-6 & Price Canyon & -\$5,966 & -\$152 & \$62,556 & \$62,404\\
SR-101 & Hyrum & -\$6,020 & -\$3 & \$67 & \$65\\
US-40 & East of Strawberry Reservoir & -\$6,036 & -\$197 & \$56,200 & \$56,003\\
\addlinespace
I-70 & Colorado state line & -\$6,056 & \$516 & \$1,603,381 & \$1,603,896\\
I-70 & East of Cove Fort & -\$6,099 & -\$13 & \$118,387 & \$118,374\\
US-89 & Arizona state line & -\$6,112 & \$315 & \$98,484 & \$98,798\\
SR-153 & Beaver Canyon & -\$6,124 & -\$120 & \$2,665 & \$2,545\\
SR-24 & West of Hanksville & -\$6,141 & -\$127 & \$227 & \$100\\
\addlinespace
I-70 & West of Green River & -\$6,157 & -\$166 & \$320,401 & \$320,235\\
SR-24 & in Capitol Reef National Park & -\$6,172 & -\$179 & \$328 & \$149\\
SR-95 & Hite & -\$6,173 & -\$187 & \$859 & \$672\\
US-6 & King Top & -\$6,173 & -\$184 & \$423 & \$239\\
SR-65 & Emigration Canyon & -\$6,173 & -\$186 & \$82 & -\$104\\
\addlinespace
US-6 & Spanish Fork Canyon & -\$7,472 & \$2,213 & \$178,115 & \$180,328\\
MVC (UT-85) & West Jordan & -\$8,406 & \$10,854 & \$329 & \$11,184\\
I-215 & Cottonwood Heights & -\$8,901 & \$34,851 & \$1,917 & \$36,768\\
SR-191 & between Helper \& Duchesne & -\$9,487 & \$31 & \$17,093 & \$17,124\\
Bangerter & near Bluffdale & -\$27,720 & \$42,503 & \$1,026 & \$43,529\\
\addlinespace
I-80 & SLC 1300 E & -\$42,433 & \$50,471 & \$41,831 & \$92,302\\
\bottomrule
\end{tabular}}
\end{table}

The remaining columns of Table~\ref{tbl-traveltimerank} present the
costs associated with link closure based on the travel time method. Many
of the most costly scenarios in the logsum model also appear to be
costly in the comparable elements of the travel time method. That is,
the scenario breaking the interstate link between Salt Lake and Utah
counties is the most costly scenario in both methods, and underscores
the significance of this link to Utah's economy and people. But while
many of the largest and most impactful scenarios have similar rankings
and scales, there are also drastic differences between the two methods
down the line. To put it simply, the choice of analysis method would
change the priority that UDOT places on its roads in terms of preparing
for incidents and hardening assets.

\hypertarget{sensitivity-analysis}{%
\subsection{Sensitivity Analysis}\label{sensitivity-analysis}}

A primary limitation of the model framework presented to this point is
is that the input parameters used for the mode and destination choice
utilities were gathered from several different sources including a
statewide trip-based model, a statewide activity-based model, and an
urban model for a small region. How much would the findings presented to
this point change if the parameters in Table~\ref{tbl-coeffs} were to
change modestly?

To examine this possibility, we construct 25 independent draws of the
parameter coefficients using Latin Hypercube Sampling (\emph{34}). The
coefficient of variation for each parameter was assumed to be 0.1;
originally, a value of 0.3 was selected (\emph{35}), but this resulted
in an unreasonable range of implied values of time. Using each of the 25
draws, we ran the base scenario and three large-impact scenarios and
calculated the logsum-based costs.

Figure~\ref{fig-saplot} presents the estimated monetary costs for each
of these three scenarios under each of the 25 parameter draws. The
results are ordered in the figure by the estimated cost for the
highest-impact scenario. Two observations can be made from this figure.
First, different parameter values do not affect the scenarios uniformly.
The second observation is that despite the within-scenario variation,
the overall scale of the three scenarios is maintained regardless of the
drawn parameter values. Indeed, the three scenarios remain in their
priority ranking across all 25 draws of the choice model parameters. We
therefore do not expect that the selection of parameters is a major
element in the relative ranking of scenarios.

\begin{figure}

{\centering \includegraphics{04-results_files/figure-pdf/fig-saplot-1.pdf}

}

\caption{\label{fig-saplot}Estimated logsum-based scenario costs in 25
different draws of the choice model parameters,}

\end{figure}

\bookmarksetup{startatroot}

\hypertarget{limitations-and-discussion}{%
\section{Limitations and Discussion}\label{limitations-and-discussion}}

The issue of systemic resilience to natural and other hazards is an
important question for UDOT and many other agencies in the United States
and around the world, given the precarious situation of many of its
transportation assets. Flash floods, rockfalls, avalanches, earthquakes,
and other incidents pose threats to the network of independent
transportation facilities. This research did not consider risk
assessment directly at any level, but rather took as a given that 41
facilities were at some level of risk and played a potentially large
systemic role. A well-conceived approach to systemic resilience should
involve all three elements of resilience: hardening assets from failure
through high-quality engineering and construction; locating maintenance
resources in areas where they can most quickly resolve issues and return
facilities optimal conditions; and understanding how the system could
work effectively in a damaged or degraded state for medium to long
periods of time if necessary.

This research focused on the systemic criticality of 41 facilities that
were assumed to fail independently. Some of the disaster scenarios most
likely to affect highway facilities -- especially a major earthquake --
are likely to damage multiple highway assets simultaneously. This
research might be extended to consider what would happen if a set of
highway facilities failed; is there a facility that is not critical were
it to fail by itself, but ends up being a critical component of several
combinations of failures? Taking the question further, agencies might
consider scenarios where emergency services or evacuations must operate
on a degraded network, an element of the emerging research area of
functional recovery (\emph{36}). Of course, it is also an open question
as to whether destination and mode choices will follow the same
behavioral patterns in such a scenario.

This research developed a trip-based statewide transportation planning
model using model design practices -- including a destination choice
trip distribution model and a complete if rudimentary logit-based mode
choice model. The model presented in this research should not be used
for infrastructure policy or forecasting analyses, but is instead an
illustrative tool. Many statewide models, on the other hand, is a
trip-based statewide transportation planning model with a basic
gravity-type trip distribution model and no mode choice component. Using
logit-based choice frameworks for trip distribution and mode choice
allows the model to incorporate greater sensitivity to influencing
variables and other benefits. It is recommended that agencies include a
logit-based mode and destination choice model as a bare minimum, if not
a full activity-based model as is becoming the standard practice.

The flexibility and sensitivity afforded by choice models can introduce
some additional challenges, however. The results of the logsum-based
choice analysis highlight one such risk, in that using different cost
functions for network skimming and destination choice can lead to
inconsistent model behavior. In this research and in the USTM highway
assignment, vehicle trips between origins and destinations are loaded
onto the shortest time path, but their destinations are chosen by a
combination of travel time and path distance. Though this issue does not
always arise during calibration and typical volume forecasting efforts,
it represents a serious inconsistency in the travel model framework.
Along the same lines, this research included only a single feedback
iteration; understanding how many iterations are necessary for a travel
model to successfully converge -- where the destinations chosen are no
longer changing based on changes to travel times between origins and
destinations -- is an important model design decision that was not
explored here.

A potential limitation in the model developed for this research is that
all HBW trips are flexible in destination choice. This implies that a
user could choose to work in a different place when the path to their
previously-chosen work location is disrupted. This might not be entirely
logical for short-term or even medium-term highway closures, considering
that most people will not switch jobs so quickly. With the recent uptick
in telecommuting instigated by the COVID-19 pandemic, workplace location
is likely even more flexible now than it has been in the past. This
increased flexibility has already called into question how HBW trips are
handled in travel behavior modeling (\emph{37}). A more nuanced method
for estimating HBW trips that accounts for both flexibility and
inflexibility of workplace location might be developed.

Policies that result in a clear and certain outcome rare, though travel
demand models are sometimes mis-interpreted, mis-used, or even
mis-designed to imply a single policy prediction. Along the same lines,
UDOT should take a proactive role in helping travel modelers and
transportation planners incorporate uncertainty in their analysis and
convey this uncertainty responsibly in communications with
transportation decision-makers and the general public.

\bookmarksetup{startatroot}

\hypertarget{conclusions}{%
\section{Conclusions}\label{conclusions}}

The question of systemic resilience of highway assets is an increasing
concern to agencies that must maintain critical infrastructure as it
ages in the face of a changing climate and economy. The basic tools of
travel forecasting --- based on coherent representations of human
behavior --- provide compelling tools for evaluating the criticality of
individual highway assets. These tools may require different expertise
than more traditional agency engineers are usually equipped with, but
simplified or simplistic methods can lead to fundamentally different
evaluations of what may happen when the network changes. Demand modelers
must first equip their models with the tools to be useful outside of
simple volume forecasting, and then communicate with their stakeholders
and peers the purpose of the model and its powerful potential for
helpful analysis.

\bookmarksetup{startatroot}

\hypertarget{acknowledgments}{%
\section*{Acknowledgments}\label{acknowledgments}}
\addcontentsline{toc}{section}{Acknowledgments}

\markboth{Acknowledgments}{Acknowledgments}

This study was funded by the Utah Department of Transportation. The
authors alone are responsible for the preparation and accuracy of the
information, data, analysis, discussions, recommendations, and
conclusions presented herein. The contents do not necessarily reflect
the views, opinions, endorsements, or policies of the Utah Department of
Transportation or the US Department of Transportation. The Utah
Department of Transportation makes no representation or warranty of any
kind, and assumes no liability therefore.

\bookmarksetup{startatroot}

\hypertarget{author-contribution-statement}{%
\section*{Author Contribution
Statement}\label{author-contribution-statement}}
\addcontentsline{toc}{section}{Author Contribution Statement}

\markboth{Author Contribution Statement}{Author Contribution Statement}

The authors confirm contribution to the paper as follows: study
conception and design: G.S. Macfarlane; data collection: M. Barnes and
N. Gray; analysis and interpretation of results: G.S. Macfarlane, M.
Barnes, and N. Gray; draft manuscript preparation: G.S. Macfarlane, M.
Barnes, and N. Gray. All authors reviewed the results and approved the
final version of the manuscript.

\bookmarksetup{startatroot}

\hypertarget{references}{%
\section*{References}\label{references}}
\addcontentsline{toc}{section}{References}

\markboth{References}{References}

\hypertarget{refs}{}
\begin{CSLReferences}{0}{0}
\leavevmode\vadjust pre{\hypertarget{ref-bradley2007}{}}%
\CSLLeftMargin{1. }%
\CSLRightInline{Bradley, J. Time Period and Risk Measures in the General
Risk Equation. \emph{Journal of Risk Research}, Vol. 10, No. 3, 2007,
pp. 355--369. \url{https://doi.org/10.1080/13669870701252232}.}

\leavevmode\vadjust pre{\hypertarget{ref-peeta2010}{}}%
\CSLLeftMargin{2. }%
\CSLRightInline{Peeta, S., F. Sibel Salman, D. Gunnec, and K. Viswanath.
Pre-Disaster Investment Decisions for Strengthening a Highway Network.
\emph{Computers \& Operations Research}, Vol. 37, No. 10, 2010, pp.
1708--1719. \url{https://doi.org/10.1016/j.cor.2009.12.006}.}

\leavevmode\vadjust pre{\hypertarget{ref-zhang2016}{}}%
\CSLLeftMargin{3. }%
\CSLRightInline{Zhang, W., and N. Wang. Resilience-Based Risk Mitigation
for Road Networks. \emph{Structural Safety}, Vol. 62, 2016, pp. 57--65.
\url{https://doi.org/10.1016/j.strusafe.2016.06.003}.}

\leavevmode\vadjust pre{\hypertarget{ref-berdica2002}{}}%
\CSLLeftMargin{4. }%
\CSLRightInline{Berdica, K. An Introduction to Road Vulnerability: What
Has Been Done, Is Done and Should Be Done. \emph{Transport Policy}, Vol.
9, No. 2, 2002, pp. 117--127.
\url{https://doi.org/10.1016/S0967-070X(02)00011-2}.}

\leavevmode\vadjust pre{\hypertarget{ref-ip2011}{}}%
\CSLLeftMargin{5. }%
\CSLRightInline{Ip, W. H., and D. Wang. Resilience and Friability of
Transportation Networks: Evaluation, Analysis and Optimization.
\emph{IEEE Systems Journal}, Vol. 5, No. 2, 2011, pp. 189--198.
\url{https://doi.org/10.1109/JSYST.2010.2096670}.}

\leavevmode\vadjust pre{\hypertarget{ref-aem2017}{}}%
\CSLLeftMargin{6. }%
\CSLRightInline{AEM. I-15 Corridor Risk and Resilience Pilot Final
Report., 2017.}

\leavevmode\vadjust pre{\hypertarget{ref-zhu2010}{}}%
\CSLLeftMargin{7. }%
\CSLRightInline{Zhu, S., D. Levinson, H. X. Liu, and K. Harder. The
Traffic and Behavioral Effects of the i-35W Mississippi River Bridge
Collapse. \emph{Transportation Research Part A: Policy and Practice},
Vol. 44, No. 10, 2010, pp. 771--784.
\url{https://doi.org/10.1016/j.tra.2010.07.001}.}

\leavevmode\vadjust pre{\hypertarget{ref-levinson2010}{}}%
\CSLLeftMargin{8. }%
\CSLRightInline{Zhu, S., D. Levinson, H. Liu, K. Harder, and A. Dancyzk.
Traffic Flow and Road User Impacts of the Collapse of the i-35W Bridge
over the Mississippi River. 2010.}

\leavevmode\vadjust pre{\hypertarget{ref-xie2011}{}}%
\CSLLeftMargin{9. }%
\CSLRightInline{Xie, F., and D. Levinson. Evaluating the Effects of the
i-35W Bridge Collapse on Road-Users in the Twin Cities Metropolitan
Region. \emph{Transportation Planning and Technology}, Vol. 34, No. 7,
2011, pp. 691--703. \url{https://doi.org/10.1080/03081060.2011.602850}.}

\leavevmode\vadjust pre{\hypertarget{ref-hamedi2018}{}}%
\CSLLeftMargin{10. }%
\CSLRightInline{Hamedi, M., S. Eshragh, M. Franz, and P. M. Sekula.
\emph{Analyzing Impact of i-85 Bridge Collapse on Regional Travel in
Atlanta}. 2018.}

\leavevmode\vadjust pre{\hypertarget{ref-abdel2007}{}}%
\CSLLeftMargin{11. }%
\CSLRightInline{Abdel-Rahim, A., P. Oman, B. K. Johnson, and R. A.
Sadiq. \href{https://doi.org/10.1109/ITSC.2007.4357801}{Assessing
Surface Transportation Network Component Criticality: A Multi-Layer
Graph-Based Approach}. 2007.}

\leavevmode\vadjust pre{\hypertarget{ref-agarwal2011}{}}%
\CSLLeftMargin{12. }%
\CSLRightInline{Agarwal, J., M. Liu, and D. Blockley.
\href{https://doi.org/10.1061/41170(400)28}{A Systems Approach to
Vulnerability Assessment}. In \emph{Vulnerability, uncertainty, and
risk}, pp. 230--237.}

\leavevmode\vadjust pre{\hypertarget{ref-zhang2015}{}}%
\CSLLeftMargin{13. }%
\CSLRightInline{Zhang, X., E. Miller-Hooks, and K. Denny. Assessing the
Role of Network Topology in Transportation Network Resilience.
\emph{Journal of Transport Geography}, Vol. 46, 2015, pp. 35--45.
\url{https://doi.org/10.1016/j.jtrangeo.2015.05.006}.}

\leavevmode\vadjust pre{\hypertarget{ref-vodak2019}{}}%
\CSLLeftMargin{14. }%
\CSLRightInline{Vodák, R., M. Bíl, T. Svoboda, Z. Křivánková, J.
Kubeček, T. Rebok, and P. Hliněný. A Deterministic Approach for Rapid
Identification of the Critical Links in Networks. \emph{Plos One}, Vol.
14, No. 7, 2019. \url{https://doi.org/10.1371/journal.pone.0219658}.}

\leavevmode\vadjust pre{\hypertarget{ref-guze2014}{}}%
\CSLLeftMargin{15. }%
\CSLRightInline{Guze, S. Graph Theory Approach to Transportation Systems
Design and Optimization. \emph{TransNav, the International Journal on
Marine Navigation and Safety of Sea Transportation}, Vol. 8, No. 4,
2014, pp. 571--578. \url{https://doi.org/10.12716/1001.08.04.12}.}

\leavevmode\vadjust pre{\hypertarget{ref-osei2014}{}}%
\CSLLeftMargin{16. }%
\CSLRightInline{Osei-Asamoah, A., and N. E. Lownes. Complex Network
Method of Evaluating Resilience in Surface Transportation Networks.
\emph{Transportation Research Record}, Vol. 2467, No. 1, 2014, pp.
120--128. \url{https://doi.org/10.3141/2467-13}.}

\leavevmode\vadjust pre{\hypertarget{ref-berdica2007}{}}%
\CSLLeftMargin{17. }%
\CSLRightInline{Berdica, K., and L.-G. Mattsson. Vulnerability: A
Model-Based Case Study of the Road Network in Stockholm. In
\emph{Critical infrastructure}, Springer, pp. 81--106.}

\leavevmode\vadjust pre{\hypertarget{ref-jaller2015}{}}%
\CSLLeftMargin{18. }%
\CSLRightInline{Jaller, M., C. A. G. Calderón, W. F. Yushimito, and I.
D. S. Díaz. An Investigation of the Effects of Critical Infrastructure
on Urban Mobility in the City of Medellín. \emph{International Journal
of Critical Infrastructures}, Vol. 11, No. 3, 2015, p. 213.
\url{https://doi.org/10.1504/ijcis.2015.072158}.}

\leavevmode\vadjust pre{\hypertarget{ref-ganin2017}{}}%
\CSLLeftMargin{19. }%
\CSLRightInline{Ganin, A. A., M. Kitsak, D. Marchese, J. M. Keisler, T.
Seager, and I. Linkov. Resilience and Efficiency in Transportation
Networks. \emph{Science advances}, Vol. 3, No. 12, 2017, p. e1701079.}

\leavevmode\vadjust pre{\hypertarget{ref-ibrahim2011}{}}%
\CSLLeftMargin{20. }%
\CSLRightInline{Ibrahim, S., R. Ammar, S. Rajasekaran, N. Lownes, Q.
Wang, and D. Sharma. \href{https://doi.org/10.1109/ISCC.2011.5983988}{An
Efficient Heuristic for Estimating Transportation Network
Vulnerability}. 2011.}

\leavevmode\vadjust pre{\hypertarget{ref-serulle2011}{}}%
\CSLLeftMargin{21. }%
\CSLRightInline{Serulle, N. U., K. Heaslip, B. Brady, W. C. Louisell,
and J. Collura. Resiliency of Transportation Network of Santo Domingo,
Dominican Republic: Case Study. \emph{Transportation Research Record},
Vol. 2234, No. 1, 2011, pp. 22--30.
\url{https://doi.org/10.3141/2234-03}.}

\leavevmode\vadjust pre{\hypertarget{ref-xu2015}{}}%
\CSLLeftMargin{22. }%
\CSLRightInline{Xu, X., A. Chen, S. Jansuwan, K. Heaslip, and C. Yang.
Modeling Transportation Network Redundancy. \emph{Transportation
research procedia}, Vol. 9, 2015, pp. 283--302.}

\leavevmode\vadjust pre{\hypertarget{ref-omer2013}{}}%
\CSLLeftMargin{23. }%
\CSLRightInline{Omer, M., A. Mostashari, and R. Nilchiani. Assessing
Resilience in a Regional Road-Based Transportation Network.
\emph{International Journal of Industrial and Systems Engineering}, Vol.
13, No. 4, 2013, pp. 389--408.
\url{https://doi.org/10.1504/ijise.2013.052605}.}

\leavevmode\vadjust pre{\hypertarget{ref-ben-akiva1985}{}}%
\CSLLeftMargin{24. }%
\CSLRightInline{Ben-Akiva, M., and S. R. Lerman.
\emph{\href{https://www.jstor.org/stable/1391567?origin=crossref}{Discrete
Choice Analysis: Theory and Applications to Travel Demand}}. MIT Press,
1985.}

\leavevmode\vadjust pre{\hypertarget{ref-geurs2004}{}}%
\CSLLeftMargin{25. }%
\CSLRightInline{Geurs, K. T., and B. van Wee. Accessibility Evaluation
of Land-Use and Transport Strategies: Review and Research Directions.
\emph{Journal of Transport Geography}, Vol. 12, No. 2, 2004, pp.
127--140. \url{https://doi.org/10.1016/j.jtrangeo.2003.10.005}.}

\leavevmode\vadjust pre{\hypertarget{ref-handy1997}{}}%
\CSLLeftMargin{26. }%
\CSLRightInline{Handy, S. L., and D. A. Niemeier. Measuring
Accessibility: An Exploration of Issues and Alternatives.
\emph{Environment and Planning A: Economy and Space}, Vol. 29, No. 7,
1997, pp. 1175--1194. \url{https://doi.org/10.1068/a291175}.}

\leavevmode\vadjust pre{\hypertarget{ref-dong2006}{}}%
\CSLLeftMargin{27. }%
\CSLRightInline{Dong, X., M. E. Ben-Akiva, J. L. Bowman, and J. L.
Walker. Moving from Trip-Based to Activity-Based Measures of
Accessibility. \emph{Transportation Research Part A: Policy and
Practice}, Vol. 40, No. 2, 2006, pp. 163--180.
\url{https://doi.org/10.1016/j.tra.2005.05.002}.}

\leavevmode\vadjust pre{\hypertarget{ref-geurs2010}{}}%
\CSLLeftMargin{28. }%
\CSLRightInline{Geurs, K., B. Zondag, G. de Jong, and M. de Bok.
Accessibility Appraisal of Land-Use/Transport Policy Strategies: More
Than Just Adding up Travel-Time Savings. \emph{Transportation Research
Part D: Transport and Environment}, Vol. 15, No. 7, 2010, pp. 382--393.
\url{https://doi.org/10.1016/j.trd.2010.04.006}.}

\leavevmode\vadjust pre{\hypertarget{ref-taylor2008}{}}%
\CSLLeftMargin{29. }%
\CSLRightInline{Taylor, M. A. P. Critical Transport Infrastructure in
Urban Areas: Impacts of Traffic Incidents Assessed Using
Accessibility-Based Network Vulnerability Analysis. \emph{Growth and
Change}, Vol. 39, No. 4, 2008, pp. 593--616.
\url{https://doi.org/10.1111/j.1468-2257.2008.00448.x}.}

\leavevmode\vadjust pre{\hypertarget{ref-masiero2012}{}}%
\CSLLeftMargin{30. }%
\CSLRightInline{Masiero, L., and R. Maggi. Estimation of Indirect Cost
and Evaluation of Protective Measures for Infrastructure Vulnerability:
A Case Study on the Transalpine Transport Corridor. \emph{Transport
Policy}, Vol. 20, 2012, pp. 13--21.}

\leavevmode\vadjust pre{\hypertarget{ref-winkler2016}{}}%
\CSLLeftMargin{31. }%
\CSLRightInline{Winkler, C. Evaluating Transport User Benefits:
Adjustment of Logsum Difference for Constrained Travel Demand Models.
\emph{Transportation Research Record}, Vol. 2564, No. 1, 2016, pp.
118--126. \url{https://doi.org/10.3141/2564-13}.}

\leavevmode\vadjust pre{\hypertarget{ref-koppelman2006}{}}%
\CSLLeftMargin{32. }%
\CSLRightInline{Koppelman, F., and C. Bhat. A Self Instructing Course in
Mode Choice Modeling: Multinomial and Nested Logit Models. 2006.}

\leavevmode\vadjust pre{\hypertarget{ref-UDOT2020}{}}%
\CSLLeftMargin{33. }%
\CSLRightInline{Transportation, U. D. of. UDOT Asset Risk Management
Process., 2020.}

\leavevmode\vadjust pre{\hypertarget{ref-helton2003}{}}%
\CSLLeftMargin{34. }%
\CSLRightInline{Helton, J. C., and F. J. Davis. Latin Hypercube Sampling
and the Propagation of Uncertainty in Analyses of Complex Systems.
\emph{Reliability Engineering \& System Safety}, Vol. 81, No. 1, 2003,
pp. 23--69. \url{https://doi.org/10.1016/S0951-8320(03)00058-9}.}

\leavevmode\vadjust pre{\hypertarget{ref-zhao2002}{}}%
\CSLLeftMargin{35. }%
\CSLRightInline{Zhao, Y., and K. M. Kockelman. The Propagation of
Uncertainty Through Travel Demand Models: An Exploratory Analysis.
\emph{The Annals of Regional Science}, Vol. 36, No. 1, 2002, pp.
145--163. \url{https://doi.org/10.1007/s001680200072}.}

\leavevmode\vadjust pre{\hypertarget{ref-zhan2022}{}}%
\CSLLeftMargin{36. }%
\CSLRightInline{Zhan, S., A. Chang-Richards, K. Elwood, and M. Boston.
\href{https://repo.nzsee.org.nz/xmlui/handle/nzsee/2507}{Post-Earthquake
Functional Recovery: A Critical Review}. 2022.}

\leavevmode\vadjust pre{\hypertarget{ref-salon2021}{}}%
\CSLLeftMargin{37. }%
\CSLRightInline{Salon, D., M. W. Conway, D. Capasso da Silva, R. S.
Chauhan, S. Derrible, A. (Kouros). Mohammadian, S. Khoeini, N. Parker,
L. Mirtich, A. Shamshiripour, E. Rahimi, and R. M. Pendyala. The
Potential Stickiness of Pandemic-Induced Behavior Changes in the United
States. \emph{Proceedings of the National Academy of Sciences}, Vol.
118, No. 27, 2021, p. e2106499118.
\url{https://doi.org/10.1073/pnas.2106499118}.}

\end{CSLReferences}



\end{document}
